\documentclass[a4paper,10pt]{article}
\usepackage{tabularx}
\usepackage{amssymb}
\usepackage{enumitem}
\usepackage{algorithm2e}
\include{../Guias/AEDmacros.tex}
\usepackage{graphicx}
\graphicspath{./}
\usepackage{changepage}
\newcommand{\tbft}[2]{\par\addvspace{\baselineskip}\textbf{#1}\hspace{0.35em}{#2}\\\par\addvspace{\baselineskip}}
\newcommand{\ejercicio}[2]{\par\addvspace{\baselineskip}\textbf{Ejercicio #1.}\hspace{0.35em}{#2}\\\par\addvspace{\baselineskip}}
% \ejercicio{NUMERO}{ENUNCIADO}
%   Devuelve
% Ejercicio NUMERO. ---- ENUNCIADO ----
%
%salto de linea comodo
\newcommand{\salto}[1]{\par\addvspace{#1}}
%
% si y solo si corto y largo
\newcommand{\sii}{\leftrightarrow}
\newcommand{\siiLargo}{\longleftrightarrow}
%
% entorno TAD
\newenvironment{tad}[1]{
%
%   constructor del encabezado formateado
\newcommand{\encabezadoTAD}[1]{\salto{1ex}\par\noindent TAD\ \ \normalfont\ttfamily#1 }
%
\encabezadoTAD{#1}\{
    \begin{adjustwidth}{3em}{0em}
    \newcommand{\obs}[2]{\par\noindent{\ttfamily obs} ##1: ##2\par}
    \newcommand{\nombretad}{{\ttfamily#1}}}
{\end{adjustwidth}\par\}}
% \begin{tad}{nombre del tad}
%   AGREGAR UN OBSERVADOR
%     \obs{nombre observador}{tipo}
%     \nombretad <---------- DEVUELVE EL NOMBRE DEL TAD (du)
%
%   y aca se pueden usar todos los procs y cosas de la catedra
%
%     \end{tad}
\begin{document}
\paragraph*{Arboles}
\begin{itemize}
    \item Estructura de datos, ramas
    \item Se puede definir recursivamente
    \item Puede ser un unico elemento, la nada, o una estructura de multiples nodos de los cuales salen multiples ramas
\end{itemize}
Vamos a usar arboles binarios, con solo dos ramas por nodo que pueden o no ser vacias, pero tienen que estar definidas.
(un poco invent eso de que tiene que estar definidas).
\salto{\baselineskip}
En general los arboles se definen en relacion a la cantidad de subarboles que pueden soportar, por ejemplo un arbol unario es una lista, un arbol binario soporta hasta dos subarboles, y asi.
\paragraph*{Propiedades interesantes de los arboles}
\begin{itemize}
    \item Cantidad de nodos
    \item Cuanto crecen vertical y horizontalmente
\end{itemize}
En haskell: (osea matematica y recursivamente)
\begin{itemize}
    \item altura(nil)=0
    \item altura(\smm{n,i,d})=1+max\{altura i, altura d\}
    \item elementos(nil)=[]
    \item elementos(\smm{n,i,d})=[n] ++ elementos i ++ elementos d
\end{itemize}
Una implementacion del arbol binario se puede implementar de una forma similar a una lista doblemente enlazada, de la forma \emph{|puntero1|valor|puntero2|}\\
Viendo a los arboles como tads, que operaciones deberia tener?
\begin{tad}{ArbolBinario}
    \obs{}{}
    \begin{proc}{vacio}{Parametros}{Resultado}
        \requiere{true}
        \asegura{true}
    \end{proc}
    \begin{proc}{altura}{Parametros}{Resultado}
        \requiere{true}
        \asegura{true}
    \end{proc}
    \begin{proc}{bin}{Parametros}{Resultado}
        \requiere{true}
        \asegura{true}
    \end{proc}
    Como el tad de arbol binario no es muy util, directamente vamos a trabajar sobre la implementacion en nodos
\end{tad}

\paragraph*{Tipo especial de arbol: Arbol Binario de Busqueda} Para todo nodo, los valores de su subarbol izquierdo son menores que el valor del nodo y los valores del subarbol derecho son mayores.
\begin{design}{ConjuntoConABB}{ArbolBinario}
    \var{e}{ArbolBinario}
\invrep{e}{esABB(e)=true}
\abs{e}{c.elems=\{n:N | n in elementos(e.raiz)\}}
\end{design}
\anotacion[red]{\Large $\uparrow$ tomar con pinzas porque la slide tiene mas typos que otra cosa }
La idea importante es la del principio, a la izquierda todos menores, a la derecha todos mayores, para cualquier nodo que agarre.
\paragraph*{Algoritmos para Arbol de Busqueda Binario}
\begin{itemize}
    \item vacio
    \item Busqueda
    \item Insercion
    \item Eliminar
\end{itemize}
Nodo = Struct \smm{dato: N, izq: Nodo, dar: Nodo} (opcionalmente un padre)
\begin{design}{AB}{Conjunto}
    \var{raiz}{Nodo}
\invrep{args}{cuerpo}
\abs{args}{cuerpo}
\begin{impl}{vacio}{}{ABB}
a= new ABB();
\\a.raiz=null;
\\return a;
\end{impl}
\begin{impl}{Busqueda}{a: ABB, k: N}{}
    return busqueda(a.raiz, k)
\end{impl}
\begin{impl}{busqueda}{n,k}{nodo}
    \ifthel{n==null || k==n.dato}{return n}{skip}
    \ifthel{k<n.dato}{busqueda(n.izq, k)}{busqueda(n.der, k)}
\end{impl}
\begin{impl}{insercion\anotacionns[red]{\ mia}}{a: ABB, n: \nat}{ABB}
    \ifthel{a.dato==n}{skip}{\ifthel{a.dato<n}{insercion(a.izq, n)}{insercion(a.der, n)}}
\\a.dato=new Nodo(null,null,k)
{\color{red}medio pelo quedo, esta mal}
\end{impl}
\pagebreak
\begin{impl}{insercion}{inout a: ABB, k: \nat}{}
    n=a.raiz\\
    padre= raiz;
    \while{a!=null}{
        padre=n;
        \ifthel{k<n.dato}{n = n.izq}{n = n.der}
    }
    newnodo= new Nodo(k, nil,nil, padre)
    \ifthel{padre==null}{a.raiz=newnodo}{\ifthel{k<padre.dato}{padre.izq=newnodo}{padre.der=newnodo}}
\end{impl}

\end{design}
\end{document}