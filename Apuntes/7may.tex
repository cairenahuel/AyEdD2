\documentclass[10pt,a4paper]{article}
\usepackage{tabularx}
\usepackage{amssymb}
\usepackage{enumitem}
\usepackage{algorithm2e}
\usepackage[spanish,activeacute,es-tabla]{babel}
\usepackage[utf8]{inputenc}
\usepackage{ifthen}
\usepackage{listings}
\usepackage{dsfont}
\usepackage{subcaption}
\usepackage{amsmath}
\usepackage{changepage}
\usepackage[top=1cm,bottom=2cm,left=1cm,right=1cm]{geometry}%
\usepackage{color}%
\newcommand{\tocarEspacios}{%
	\addtolength{\leftskip}{3em}%
	\setlength{\parindent}{0em}%
}

% Especificacion de procs

\newcommand{\In}{\textsf{in }}
\newcommand{\Out}{\textsf{out }}
\newcommand{\Inout}{\textsf{inout }}

\newcommand{\encabezadoDeProc}[4]{%
	% Ponemos la palabrita problema en tt
	%  \noindent%
	{\normalfont\bfseries\ttfamily proc}%
	% Ponemos el nombre del problema
	\ %
	{\normalfont\ttfamily #2}%
	\
	% Ponemos los parametros
	(#3)%
	\ifthenelse{\equal{#4}{}}{}{%
		% Por ultimo, va el tipo del resultado
		\ : #4}
}

\newenvironment{proc}[4][res]{%
	
	% El parametro 1 (opcional) es el nombre del resultado
	% El parametro 2 es el nombre del problema
	% El parametro 3 son los parametros
	% El parametro 4 es el tipo del resultado
	% Preambulo del ambiente problema
	% Tenemos que definir los comandos requiere, asegura, modifica y aux
	\newcommand{\requiere}[2][]{%
		{\normalfont\bfseries\ttfamily requiere}%
		\ifthenelse{\equal{##1}{}}{}{\ {\normalfont\ttfamily ##1} :}\ %
		\{\ensuremath{##2}\}%
		{\normalfont\bfseries\,\par}%
	}
	\newcommand{\asegura}[2][]{%
		{\normalfont\bfseries\ttfamily asegura}%
		\ifthenelse{\equal{##1}{}}{}{\ {\normalfont\ttfamily ##1} :}\
		\{\ensuremath{##2}\}%
		{\normalfont\bfseries\,\par}%
	}
	\renewcommand{\aux}[4]{%
		{\normalfont\bfseries\ttfamily aux\ }%
		{\normalfont\ttfamily ##1}%
		\ifthenelse{\equal{##2}{}}{}{\ (##2)}\ : ##3\, = \ensuremath{##4}%
		{\normalfont\bfseries\,;\par}%
	}
	\renewcommand{\pred}[3]{%
		{\normalfont\bfseries\ttfamily pred }%
		{\normalfont\ttfamily ##1}%
		\ifthenelse{\equal{##2}{}}{}{\ (##2) }%
		\{%
		\begin{adjustwidth}{+5em}{}
			\ensuremath{##3}
		\end{adjustwidth}
		\}%
		{\normalfont\bfseries\,\par}%
	}
	
	\newcommand{\res}{#1}
	\vspace{1ex}
	\noindent
	\encabezadoDeProc{#1}{#2}{#3}{#4}
	% Abrimos la llave
	\par%
	\tocarEspacios
}
{
	% Cerramos la llave
	\vspace{1ex}
}

\newcommand{\aux}[4]{%
	{\normalfont\bfseries\ttfamily\noindent aux\ }%
	{\normalfont\ttfamily #1}%
	\ifthenelse{\equal{#2}{}}{}{\ (#2)}\ : #3\, = \ensuremath{#4}%
	{\normalfont\bfseries\,;\par}%
}

\newcommand{\pred}[3]{%
	{\normalfont\bfseries\ttfamily\noindent pred }%
	{\normalfont\ttfamily #1}%
	\ifthenelse{\equal{#2}{}}{}{\ (#2) }%
	\{%
	\begin{adjustwidth}{+2em}{}
		\ensuremath{#3}
	\end{adjustwidth}
	\}%
	{\normalfont\bfseries\,\par}%
}

% Tipos

\newcommand{\nat}{\ensuremath{\mathds{N}}}
\newcommand{\ent}{\ensuremath{\mathds{Z}}}
\newcommand{\float}{\ensuremath{\mathds{R}}}
\newcommand{\bool}{\ensuremath{\mathsf{Bool}}}
\newcommand{\cha}{\ensuremath{\mathsf{Char}}}
\newcommand{\str}{\ensuremath{\mathsf{String}}}

% Logica

\newcommand{\True}{\ensuremath{\mathrm{true}}}
\newcommand{\False}{\ensuremath{\mathrm{false}}}
\newcommand{\Then}{\ensuremath{\rightarrow}}
\newcommand{\Iff}{\ensuremath{\leftrightarrow}}
\newcommand{\implica}{\ensuremath{\longrightarrow}}
\newcommand{\IfThenElse}[3]{\ensuremath{\mathsf{if}\ #1\ \mathsf{then}\ #2\ \mathsf{else}\ #3\ \mathsf{fi}}}
\newcommand{\yLuego}{\land _L}
\newcommand{\oLuego}{\lor _L}
\newcommand{\implicaLuego}{\implica _L}

\newcommand{\cuantificador}[5]{%
	\ensuremath{(#2 #3: #4)\ (%
		\ifthenelse{\equal{#1}{unalinea}}{
			#5
		}{
			$ % exiting math mode
			\begin{adjustwidth}{+2em}{}
				$#5$%
			\end{adjustwidth}%
			$ % entering math mode
		}
		)}
}

\newcommand{\existe}[4][]{%
	\cuantificador{#1}{\exists}{#2}{#3}{#4}
}
\newcommand{\paraTodo}[4][]{%
	\cuantificador{#1}{\forall}{#2}{#3}{#4}
}

%listas

\newcommand{\TLista}[1]{\ensuremath{seq \langle #1\rangle}}
\newcommand{\lvacia}{\ensuremath{[\ ]}}
\newcommand{\lv}{\ensuremath{[\ ]}}
\newcommand{\longitud}[1]{\ensuremath{|#1|}}
\newcommand{\cons}[1]{\ensuremath{\mathsf{addFirst}}(#1)}
\newcommand{\indice}[1]{\ensuremath{\mathsf{indice}}(#1)}
\newcommand{\conc}[1]{\ensuremath{\mathsf{concat}}(#1)}
\newcommand{\cab}[1]{\ensuremath{\mathsf{head}}(#1)}
\newcommand{\cola}[1]{\ensuremath{\mathsf{tail}}(#1)}
\newcommand{\sub}[1]{\ensuremath{\mathsf{subseq}}(#1)}
\newcommand{\en}[1]{\ensuremath{\mathsf{en}}(#1)}
\newcommand{\cuenta}[2]{\mathsf{cuenta}\ensuremath{(#1, #2)}}
\newcommand{\suma}[1]{\mathsf{suma}(#1)}
\newcommand{\twodots}{\ensuremath{\mathrm{..}}}
\newcommand{\masmas}{\ensuremath{++}}
\newcommand{\matriz}[1]{\TLista{\TLista{#1}}}
\newcommand{\seqchar}{\TLista{\cha}}

\renewcommand{\lstlistingname}{Código}
\lstset{% general command to set parameter(s)
	language=Java,
	morekeywords={endif, endwhile, skip},
	basewidth={0.47em,0.40em},
	columns=fixed, fontadjust, resetmargins, xrightmargin=5pt, xleftmargin=15pt,
	flexiblecolumns=false, tabsize=4, breaklines, breakatwhitespace=false, extendedchars=true,
	numbers=left, numberstyle=\tiny, stepnumber=1, numbersep=9pt,
	frame=l, framesep=3pt,
	captionpos=b,
}
\usepackage{changepage}
\usepackage{xcolor}
\newcommand{\tbft}[2]{\par\addvspace{\baselineskip}\textbf{#1}\hspace{0.35em}{#2}\\\par\addvspace{\baselineskip}}
\newcommand{\ejercicio}[2]{\par\addvspace{\baselineskip}\textbf{Ejercicio #1.}\hspace{0.35em}{#2}\\\par\addvspace{\baselineskip}}
% \ejercicio{NUMERO}{ENUNCIADO}
%   Devuelve
% Ejercicio NUMERO. ---- ENUNCIADO ----
%
%formato
\newcommand{\salto}[1]{\par\addvspace{#1}}
\newcommand{\rojo}[1]{{\color{red}#1}}
\newcommand{\anotacion}[2][red]{\salto{1ex}\noindent\texttt{\color{#1}#2}\salto{1ex}}
\newcommand{\anotacionns}[2][red]{\noindent\texttt{\color{#1}#2}}
%
% si y solo si corto y largo
\newcommand{\sii}{\leftrightarrow}
\newcommand{\siiLargo}{\longleftrightarrow}
\newcommand{\slr}[1]{\ensuremath{\langle #1\rangle}}
\newcommand{\smm}[1]{\textless #1\textgreater}
\newcommand{\encabezadoTAD}[1]{\par\salto{1ex}\noindent TAD\ \ \normalfont\ttfamily#1 }
\newenvironment{tad}[1]{
\newcommand{\nombretad}{{\ttfamily#1}}
\newcommand{\nt}{\nombretad}
\newcommand{\obs}[2]{\par\noindent{\ttfamily obs} ##1: ##2\par}
\encabezadoTAD{#1}\{
    \begin{adjustwidth}{3em}{0em}}
{\end{adjustwidth}\par\}}
% \begin{tad}{nombre del tad}
%   AGREGAR UN OBSERVADOR
%     \obs{nombre observador}{tipo}
%     \nombretad <---------- DEVUELVE EL NOMBRE DEL TAD (du)
%
%   y aca se pueden usar todos los procs y cosas de la catedra
%
%     \end{tad}
\begin{document}
\section*{Repaso TADs}
\paragraph*{Que es?}
El que y no el como.
Tiene estado
Se manipula atravez de operaciones
\paragraph*{Diseño de TADs}
Un diseño es una estrctura de datos y una serie de algoritmos que nos indica como se representa.
\\- Tenemos que elegir una estructura de datos para implementarlo, con sus pros y sus contras.
\\Hay muchas formas de hacer lo mismo, las personas piensan de formas distintas y/o tienen distintos motores, (necesitas mas rapidez, eficiencia, etc)
\\Lo importante es cumplir el qué(la especificacion), el como cada uno lo maneja.
\paragraph*{Ocultar informacion}
Principio de ocultacion. No tengo por que mostrar como hago las cosas mientras se respete el resultado final.
\\Facilita la comprension.
\\Favorece el reuso.
\\Ayuda a modularizar y separar el trabajo, es tambien el sistema mas resistente a cambios.
\salto{\baselineskip}
Encapsulamiento = god
\paragraph*{TAD Punto}
Primero pienso las operaciones, los observadores solo sirven para explicar que hacen.
\\\texttt{
    modulo PuntoImpl implementa Punto
        \\var rho: float
        \\var theta: float
        \anotacion{da igual como lo hagamos, mientras haga lo que debe.}
        \anotacion{Cuando definiamos un tad usabamos tipos basicos, el mundo de la "matematica", ahora los tipos de las variables
        son los tipos de implementacion.
        \\-int, float, char...
        \\-tupla arrays
        \\estructs, modulos de otros tads, y mas cosas.}
}
\paragraph*{El invariante de representacion}
Cuando elegimos los tipos tenemos que tener cuidado de que pueda cumplir con lo que necesitamos (si queremos hacer una lista de ocho elementos, por ejemplo una tupla de, que es fija, de dos casillas no nos serviria)
\\Volviendo al ejemplo del punto:
\\Tenemos al menos dos formas de almacenar el angulo theta, normalizado (entre cero y dos pi, o entre menos pi y mas pi), desnormalizado(cualquier valor real)
\\Tenemos que ser consistentes, nuestras operaciones pueden asumir que se cumplen estas cosas, pero al final de nuestra operacion debemos asegurarnos de que todo siga siendo consistente y cumplamos nuestro propio "contrato".
\\Es algo que me dice el estado de consistencia de mi representacion interna.
\paragraph*{TAD Punto: Invariante de representacion}
El invariante se escribe en logica(usando lenguaje de especificacion) haciendo referencia a la estructura de implementacion.
\texttt{
    \\\pred{InvRep}{p':PuntoImpl}{
        -\pi\leq p'.theta<\pi \land rho\geq 0
    }}
\paragraph*{Funcion o predicado de abstraccion}Tenemos dos estrucutras, el tad y la implementacion. ¿Como los relacionamos?
\\Nos indica para cada instancia de la implementacion, a que instancia del tad representa, que instancia del tad "es su abstraccion"
\\Hace referencia a las variables de estado de la implementacion y a los observadores del TAD (porque tiene que vincular unas cosas con otras)
\\Para definir se puede suponer que vale el invariante de representacion.
\paragraph*{TAD Punto: funcion (predicado) de abstraccion}
\texttt{\\FuncAbs(p':PuntoImpl): Punto \{
\\p: punto |\\
\hspace*{2em}p.x=p'.rho*cos(p'.theta)\&\& p.y=p'.rho*sin(p'.theta)
\}}
\anotacion{Un tad es una entidad con estado que manipulas con operaciones, no te importa cuanto vale solo te importa lo que puede hacer.}
\anotacion[red]{La funcion abstraccion conecta ambos mundos. Nos explica como se mapean los observadores, en que se traducen en la realidad.}
Tambien podemos escribirla como un predicado si nos resulta mas comodo.
\anotacion[red]{Es la explicacion de como conectamos lo concreto con lo abstracto.}
\paragraph*{TAD Punto: algoritmos}Solo nos queda escribir los algoritmos:
\begin{algorithm*}
    impl mover(inout c': PuntoImpl, in deltaX: float, in deltaY: float)\{
        \\float nuevoX := c'.coordX() + deltaX;
        \\float nuevoY := c'.coordY() + deltaY;
        \BlankLine
        c'.rho := $sqrt(nuevoX^2+nuevoY^2)$;
        \\c'.theta := arctan(nuevoY / nuevoX);
    \\\}
\end{algorithm*}
\begin{impl}{hola}{prueba}
\asg{u}{cosa}
\end{impl}
\section*{TAD Conjunto: diseño}
colgue toda esta parte
\section*{Correctitud en TADs}
Verificacion: Se puede demostrar que la implementacion de un tad es correcta respecto a la especificacion.
\\El abstracto y el concreto se tienen que "mover igual"
\\Hay que demostrar que cada operacion conserva el invariante, y que el algoritmo respeta la pre y post condicion del tad.
\\agarras el concreto, abstraelo, fijate la pre, aplica la funcion, fijate la post
\salto{\baselineskip}
$preTAD \rightarrow_{abstraccion} preImpl \rightarrow postImpl \rightarrow_{abstraccion} postTAD$
\section*{TAD ConjuntoAcotado}

\end{document}