\documentclass[a4paper,10pt]{article}
\usepackage{changepage}
\newcommand{\tbft}[2]{\par\addvspace{\baselineskip}\textbf{#1}\hspace{0.35em}{#2}\\\par\addvspace{\baselineskip}}
\newcommand{\ejercicio}[2]{\par\addvspace{\baselineskip}\textbf{Ejercicio #1.}\hspace{0.35em}{#2}\\\par\addvspace{\baselineskip}}
% \ejercicio{NUMERO}{ENUNCIADO}
%   Devuelve
% Ejercicio NUMERO. ---- ENUNCIADO ----
%
%salto de linea comodo
\newcommand{\salto}[1]{\par\addvspace{#1}}
%
% si y solo si corto y largo
\newcommand{\sii}{\leftrightarrow}
\newcommand{\siiLargo}{\longleftrightarrow}
%
% entorno TAD
\newenvironment{tad}[1]{
%
%   constructor del encabezado formateado
\newcommand{\encabezadoTAD}[1]{\salto{1ex}\par\noindent TAD\ \ \normalfont\ttfamily#1 }
%
\encabezadoTAD{#1}\{
    \begin{adjustwidth}{3em}{0em}
    \newcommand{\obs}[2]{\par\noindent{\ttfamily obs} ##1: ##2\par}
    \newcommand{\nombretad}{{\ttfamily#1}}}
{\end{adjustwidth}\par\}}
% \begin{tad}{nombre del tad}
%   AGREGAR UN OBSERVADOR
%     \obs{nombre observador}{tipo}
%     \nombretad <---------- DEVUELVE EL NOMBRE DEL TAD (du)
%
%   y aca se pueden usar todos los procs y cosas de la catedra
%
%     \end{tad}
\input{../../../AEDmacros.tex}
\begin{document}
{{\LARGE Clase de Laboratorio:}\hfill {24 de Mayo}}\\
Vamos a implementar conjuntos mediante de datos eficientes.
Comlejidad de las estructuras implementadas hasta ahora:
\begin{tabular}{c|c|c|c}
&arregki red& lista enlazada & lista bi enlazada\\    
pertenece & $O(logN)$ & $O(N)$ & $O(N)$ \\
\hline\\
insertar & $O(N)$ & $O(N)$ & $O(N)$ \\
\hline\\
borrar & $O(N)$ & $O(N)$ & $O(N)$ \\
\hline\\
max/min & $O(1)$ & $O(N)/O(1)$ & $O(N)/O(1)$ \\
\end{tabular}
\paragraph*{Arboles binarios:} Cada nodo un solo padre y hasta dos hijos, hijos a la izquierda menores y a la derecha mayores.
\\Invariante de representacion de un arbol binario: 
\invrep{arbol}{un objeto es ABB $\sii$ es null $\lor$ todos los nodos a la derecha son mayores y los de la izquierda son menores, y estos tambien son arboles de busqueda}
\\El objetivo sera implementar un conjunto mediante arboles binarios de busqueda.
\paragraph*{Algoritmos\\}
\emph{abb.pertenece}
\ifthel{raiz.valor==elemento}{return true}{\ifthel{elemento$<$raiz.valor}{raiz.izquierda.pertenece(elemento)}{raiz.derecha.pertenece(elemento)}}
\\
\\\emph{abb.insertar(elem)}
Parecido al anterior
\\(porque devuelve el ultimo nodo de la busqueda)
\\Si lo encontramos no hacemos nada
\\Si NO lo encontramos lo insertamos como hijo del ultimo nodo buscado
\\
\\\emph{eliminar}
\\recorrer y si esta eliminarlo si no no hacerlo,
\\si esta y tenemos que eliminarlo, si el nodo no tiene hijos lo eliminamos directamente.
\\si tiene un unico elemento, enlazar el padre del nodo que queremos eliminar al hijo de este mismo.
\\si tiene dos hijos, buscamos el mayor de los hijos menores, o el menor de los hijos mayores, y lo eliminamos y lo reemplazamos en lugar del padre
\end{document}