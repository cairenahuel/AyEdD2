\usepackage{changepage}
\usepackage{xcolor}
\newcommand{\tbft}[2]{\par\addvspace{\baselineskip}\textbf{#1}\hspace{0.35em}{#2}\\\par\addvspace{\baselineskip}}
\newcommand{\ejercicio}[2]{\par\addvspace{\baselineskip}\textbf{Ejercicio #1.}\hspace{0.35em}{#2}\\\par\addvspace{\baselineskip}}
% \ejercicio{NUMERO}{ENUNCIADO}
%   Devuelve
% Ejercicio NUMERO. ---- ENUNCIADO ----
%
%formato
\newcommand{\salto}[1]{\par\addvspace{#1}}
\newcommand{\rojo}[1]{{\color{red}#1}}
\newcommand{\anotacion}[2][red]{\salto{1ex}\noindent\texttt{\color{#1}#2}\salto{1ex}}
\newcommand{\anotacionns}[2][red]{\noindent\texttt{\color{#1}#2}}
%
% si y solo si corto y largo
\newcommand{\sii}{\leftrightarrow}
\newcommand{\siiLargo}{\longleftrightarrow}
\newcommand{\slr}[1]{\ensuremath{\langle #1\rangle}}
\newcommand{\smm}[1]{\textless #1\textgreater}
\newcommand{\encabezadoTAD}[1]{\par\salto{1ex}\noindent TAD\ \ \normalfont\ttfamily#1 }
\newenvironment{tad}[1]{
\newcommand{\nombretad}{{\ttfamily#1}}
\newcommand{\nt}{\nombretad}
\newcommand{\obs}[2]{\par\noindent{\ttfamily obs} ##1: ##2\par}
\encabezadoTAD{#1}\{
    \begin{adjustwidth}{3em}{0em}}
{\end{adjustwidth}\par\}}
% \begin{tad}{nombre del tad}
%   AGREGAR UN OBSERVADOR
%     \obs{nombre observador}{tipo}
%     \nombretad <---------- DEVUELVE EL NOMBRE DEL TAD (du)
%
%   y aca se pueden usar todos los procs y cosas de la catedra
%
%     \end{tad}
\newcommand{\encabezadoImpl}[3]{\par\salto{1ex}\noindent \textbf{impl}\ \normalfont\ttfamily#1(#2):#3}
\newcommand{\asg}[2]{\par\noindent{\ttfamily #1}:= #2\par}
\newcommand{\asgns}[2]{{\ttfamily #1}:= #2}
\newenvironment{impl}[3]{\encabezadoImpl{#1}{#2}{#3}\{
\begin{adjustwidth}{3em}{0em}
}
{\end{adjustwidth}\par\}}
\newcommand{\encabezadoDesign}[2]{\par\salto{1ex}\noindent \textbf{modulo}\ {\normalfont\ttfamily #1} \textbf{implementa}{ \normalfont\ttfamily #2}}
\newenvironment{design}[2]{\encabezadoDesign{#1}{#2}\{
\newcommand{\var}[2]{\textbf{var} \texttt{##1}: \texttt{##2}\\}
\begin{adjustwidth}{3em}{0em}
}
{\end{adjustwidth}\par\}}
\newcommand{\ifthel}[3]{\salto{0em}\noindent\textbf{if}{ #1 }\textbf{then}\salto{0em}\noindent{\begin{adjustwidth}{1.5em}{0em}#2\end{adjustwidth}}\salto{0em}\noindent\textbf{else}\begin{adjustwidth}{1.5em}{0em}{#3}\end{adjustwidth}\salto{0em}\noindent\textbf{endif}}
\newcommand{\while}[2]{
    \salto{0em}\noindent
    \textbf{while} #1 \textbf{do}
    \begin{adjustwidth}{1.5em}{0em}{#2}\end{adjustwidth}\salto{0em}\noindent\textbf{end while}
}