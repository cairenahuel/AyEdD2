\documentclass[a4paper,10pt]{article}
\include{../../AEDmacros}
\usepackage{changepage}
\newcommand{\tbft}[2]{\par\addvspace{\baselineskip}\textbf{#1}\hspace{0.35em}{#2}\\\par\addvspace{\baselineskip}}
\newcommand{\ejercicio}[2]{\par\addvspace{\baselineskip}\textbf{Ejercicio #1.}\hspace{0.35em}{#2}\\\par\addvspace{\baselineskip}}
% \ejercicio{NUMERO}{ENUNCIADO}
%   Devuelve
% Ejercicio NUMERO. ---- ENUNCIADO ----
%
%salto de linea comodo
\newcommand{\salto}[1]{\par\addvspace{#1}}
%
% si y solo si corto y largo
\newcommand{\sii}{\leftrightarrow}
\newcommand{\siiLargo}{\longleftrightarrow}
%
% entorno TAD
\newenvironment{tad}[1]{
%
%   constructor del encabezado formateado
\newcommand{\encabezadoTAD}[1]{\salto{1ex}\par\noindent TAD\ \ \normalfont\ttfamily#1 }
%
\encabezadoTAD{#1}\{
    \begin{adjustwidth}{3em}{0em}
    \newcommand{\obs}[2]{\par\noindent{\ttfamily obs} ##1: ##2\par}
    \newcommand{\nombretad}{{\ttfamily#1}}}
{\end{adjustwidth}\par\}}
% \begin{tad}{nombre del tad}
%   AGREGAR UN OBSERVADOR
%     \obs{nombre observador}{tipo}
%     \nombretad <---------- DEVUELVE EL NOMBRE DEL TAD (du)
%
%   y aca se pueden usar todos los procs y cosas de la catedra
%
%     \end{tad}
\usepackage[dvipsnames]{xcolor}
\usepackage{graphicx}
\usepackage{microtype}
\geometry{
    a4paper,
    total={170mm,257mm},
    left=5mm,
    top=10mm,
    }
\graphicspath{{images/}}
\begin{document}
\includegraphics[width=\textwidth]{e1.png}
a) La poscondicion del ciclo sera la misma que el \emph{asegura} de la especificacion, $res=\sum_{j=0}^{i-1}s[j]$

La precondicion por otro lado no estoy seguro, sera que res=0 y i=0(las primeras dos lineas de S)?

Could be

b) de reemplazarse el primer termino del invariante este fallaria a la hora de mantenerse cierto al salir del ciclo

Esto pues el while aumenta i en uno |s| veces, donde la |s|-ava vez no entra al ciclo y el programa termina.

c) Deja de ser cierto durante toda la ejecucion del programa, basicamente te esta pidiendo que el termino que el programa esta por sumar ya este sumado a res

d) Al llegar a la ultima iteracion se intentaria acceder a un elemento del arreglo que no existe?¿? creo

f) \begin{itemize}
    \item $Pc\implies I$

          $res=0 \land i=0 \implies 0\leq i \leq |s| \land_L res=\sum_{j=0}^{i-1}s[j]$

          $res=0 \land i=0 \implies 0\leq 0 \leq |s| \land_L 0=\sum_{j=0}^{1-1}$

          $res=0 \land i=0 \implies true$

    \item $\{I\land B\}S\{I\}$

          {\color{blue}\{$0\leq i \leq |s| \land_L res=\sum_{j=0}^{i-1}s[j] \land i<|s|\}\equiv \{0\leq i < |s| \land_L res=\sum_{j=0}^{i-1}s[j]\}$}

          res:= res + s[i]

          i:=i+1

          ${\color{blue}\{0\leq i \leq |s| \land_L res=\sum_{j=0}^{i-1}s[j]\}}$

          Esto se traduce a comparar la primera parte de la tripla de Hoare con wp(res:=res+s[i]; i:=i+1, I)

          ${\color{blue}\{0\leq i<|s| \land_L 0\leq i+1 \leq |s| \land_L res+s[i]=\sum_{j=0}^{i}s[j]\}}$

          ${\color{blue}\equiv\{ 0\leq i<|s| \land_L -1\leq i \leq |s|-1 \land_L res=\sum_{j=0}^{i-1}s[j]\}}$

          ${\color{blue}\equiv\{ 0\leq i < |s| \land_L res=\sum_{j=0}^{i-1}s[j]\}}$

          Y podemos ver que ambos $I\land B$ y la wp calculada son iguales, por lo tanto la tripla de Hoare es correcta

    \item $I \land \lnot B \implies Q_c$

          $0\leq i \leq |s| \land_L res=\sum_{j=0}^{i-1}s[j] \land \lnot (i<|s|)$

          $\equiv 0\leq i \leq |s| \land_L res=\sum_{j=0}^{i-1}s[j] \land i\geq|s|$

          $\equiv i = |s| \land_L res=\sum_{j=0}^{i-1}s[j]$

          $\equiv res=\sum_{j=0}^{|s|-1}s[j]$

          Que es efectivamente nuestra postcondicion de la especificacion.
\end{itemize}
\pagebreak

f)  Propuesta 1: $fv=|s|-i+1$

\salto{\baselineskip}

$\{I\land B \land v_0=fv\} res:=res+s[i];i:=i+1 \{fv<v_0\}\equiv Pc\implies wp(s,fv<v_0)$

\salto{\baselineskip}

$wp(S,fv<v_0)\equiv wp(res:=res+s[i], wp(i:=i+1, |s|-i+1<v_0))$

cosas...

\hspace*{2em}$\equiv |s|-(i+1)+1<v_0 \equiv |s|-i<v_0$

\salto{\baselineskip}

retomando el principio:

$I\land B \land v_0=fv \implies |s|-i<v_0$?

Si, pues $ v_o=fv \implies |s|-i<|s|-i+1$ que es trivialmente cierto.

Ahora, mi propuesta cumple con el segundo statement?

$I \land fv\leq 0 \implies \lnot B$

$\alpha:\ |s|-i+1\leq 0 \implies |s|+1\leq i$

$\beta:\ \lnot B \equiv i\geq |s| \equiv |s|\leq i\hspace*{5em} \longleftarrow $A lo que quiero llegar

$I\equiv 0\leq i \leq |s| \land_L res=\sum_{j=0}^{i-1}s[j]$

Tengo informacion de sobra, $\alpha \implies |s| \leq i \equiv \lnot B$

Seems a bit too easy pero es el primer ejercicio de la guia so...


\pagebreak


\includegraphics[width=\textwidth]{e2.png}

La precondicion del ciclo seria ...........................

La poscondicion es $res=\sum_{j=0}^{n-1}if\ j\ mod 2 = 0\ then\ j\ else\ 0$

$P\implies I$ es trivialmente cierto

\salto{\baselineskip}

$2^{da})$$\{I \land B \} res:=res+i; i:=i+2\{I\}$

    \salto{\baselineskip}

$I\equiv 0\leq i\leq n+1 \land i\%2=0 \land res=\sum_{j=0}^{i-1}if\ j\%2 = 0\ then\ j\ else\ 0$ {\small {\color{Violet}$\hfill \leftarrow$Por las dudas: Uso $\%$ en lugar de mod por cuestiones de spacing}}

$B\equiv i<n$

$wp(S,I)\equiv 0\leq i\leq n+1 \land i\%2=0 \land res+i=\sum_{j=0}^{i+1}if\ j\%2 = 0\ then\ j\ else\ 0$

Analizando un poco nos damos cuenta de que i siempre sera par, por lo que i+1 es impar, entonces el termino i+1 de la sumatoria siempre sumara 0 (por el condicional), asi que podemos obviarlo. A su vez, podemos restar a ambos lados el termino i-esimo, que este sera par y por lo tanto si contara en la sumatoria, simplificando la expresion.

$wp(S,I)\equiv 0\leq i\leq n+1 \land i\%2=0 \land res=\sum_{j=0}^{i-1}if\ j\%2 = 0\ then\ j\ else\ 0$

Entonces, $I\land B \equiv 0\leq i<n \land i\%2=0 \land res=\sum_{j=0}^{i-1}if\ j\%2 = 0\ then\ j\ else\ 0$

Que efectivamente implica mi wp, pues el segundo termino es identico y en el primero,

$0\leq i<n\implies 0\leq i\leq n+1$

\salto{\baselineskip}

$3^{ra})\ I\land\lnot B\implies Q_c$

$I\equiv 0\leq i\leq n+1 \land i\%2=0 \land res=\sum_{j=0}^{i-1}if\ j\%2 = 0\ then\ j\ else\ 0$

$\lnot B\equiv i\geq n$

$n\leq i \land i\leq n+1\equiv n\leq i\leq n+1$

Pero i tiene que ser par, por lo que: $n\leq i\leq n+1\equiv n=i=n \equiv i=n$

$i=n \land i\%2=0 \land res=\sum_{j=0}^{i-1}if\ j\%2 = 0\ then\ j\ else\ 0 \equiv res=\sum_{j=0}^{n-1}if\ j\%2 = 0\ then\ j\ else\ 0 \equiv Q_c$

Posible invariante: $n-i+1$

$\{I\land B\land v_0=fv\} S \{fv<v_0\}$

$wp(i:=i+2, n-i+1<v_0)\equiv n-i-1<v_0$

entonces $v_0=fv \implies n-i-1<n-i+1$

$5^{ta})\ I\land fv\leq 0 \implies \lnot B$

Quiero entonces llegar a $\lnot B \equiv i\geq n$

$fv\leq 0 \implies n+1\leq i$

$I\implies 0\leq i\leq n+1$

Estas dos cosas implican que $i=n+1$ que a su vez implica $i\geq n$

De forma un poco confusa y poco organizada, pero queda entonces demostrada la correctitud del ciclo.
\pagebreak
\end{document}