\documentclass[a4paper,10pt]{article}
\include{../../AEDmacros}
\usepackage{changepage}
\newcommand{\tbft}[2]{\par\addvspace{\baselineskip}\textbf{#1}\hspace{0.35em}{#2}\\\par\addvspace{\baselineskip}}
\newcommand{\ejercicio}[2]{\par\addvspace{\baselineskip}\textbf{Ejercicio #1.}\hspace{0.35em}{#2}\\\par\addvspace{\baselineskip}}
% \ejercicio{NUMERO}{ENUNCIADO}
%   Devuelve
% Ejercicio NUMERO. ---- ENUNCIADO ----
%
%salto de linea comodo
\newcommand{\salto}[1]{\par\addvspace{#1}}
%
% si y solo si corto y largo
\newcommand{\sii}{\leftrightarrow}
\newcommand{\siiLargo}{\longleftrightarrow}
%
% entorno TAD
\newenvironment{tad}[1]{
%
%   constructor del encabezado formateado
\newcommand{\encabezadoTAD}[1]{\salto{1ex}\par\noindent TAD\ \ \normalfont\ttfamily#1 }
%
\encabezadoTAD{#1}\{
    \begin{adjustwidth}{3em}{0em}
    \newcommand{\obs}[2]{\par\noindent{\ttfamily obs} ##1: ##2\par}
    \newcommand{\nombretad}{{\ttfamily#1}}}
{\end{adjustwidth}\par\}}
% \begin{tad}{nombre del tad}
%   AGREGAR UN OBSERVADOR
%     \obs{nombre observador}{tipo}
%     \nombretad <---------- DEVUELVE EL NOMBRE DEL TAD (du)
%
%   y aca se pueden usar todos los procs y cosas de la catedra
%
%     \end{tad}
\usepackage[dvipsnames]{xcolor}
\usepackage{graphicx}
\usepackage{microtype}
\usepackage[spanish]{babel}
\geometry{
    a4paper,
    total={170mm,257mm},
    left=5mm,
    right=20mm,
    top=10mm,
    }
\graphicspath{{images/}}
\begin{document}
\includegraphics[width=\textwidth]{e1.png}
a) La poscondicion del ciclo sera la misma que el \emph{asegura} de la especificacion, $res=\sum_{j=0}^{i-1}s[j]$

La precondicion por otro lado no estoy seguro, sera que res=0 y i=0(las primeras dos lineas de S)?

Could be

b) de reemplazarse el primer termino del invariante este fallaria a la hora de mantenerse cierto al salir del ciclo

Esto pues el while aumenta i en uno |s| veces, donde la |s|-ava vez no entra al ciclo y el programa termina.

c) Deja de ser cierto durante toda la ejecucion del programa, basicamente te esta pidiendo que el termino que el programa esta por sumar ya este sumado a res

d) Al llegar a la ultima iteracion se intentaria acceder a un elemento del arreglo que no existe?¿? creo

f) \begin{itemize}
    \item $Pc\implies I$

          $res=0 \land i=0 \implies 0\leq i \leq |s| \land_L res=\sum_{j=0}^{i-1}s[j]$

          $res=0 \land i=0 \implies 0\leq 0 \leq |s| \land_L 0=\sum_{j=0}^{1-1}$

          $res=0 \land i=0 \implies true$

    \item $\{I\land B\}S\{I\}$

          {\color{blue}\{$0\leq i \leq |s| \land_L res=\sum_{j=0}^{i-1}s[j] \land i<|s|\}\equiv \{0\leq i < |s| \land_L res=\sum_{j=0}^{i-1}s[j]\}$}

          res:= res + s[i]

          i:=i+1

          ${\color{blue}\{0\leq i \leq |s| \land_L res=\sum_{j=0}^{i-1}s[j]\}}$

          Esto se traduce a comparar la primera parte de la tripla de Hoare con wp(res:=res+s[i]; i:=i+1, I)

          ${\color{blue}\{0\leq i<|s| \land_L 0\leq i+1 \leq |s| \land_L res+s[i]=\sum_{j=0}^{i}s[j]\}}$

          ${\color{blue}\equiv\{ 0\leq i<|s| \land_L -1\leq i \leq |s|-1 \land_L res=\sum_{j=0}^{i-1}s[j]\}}$

          ${\color{blue}\equiv\{ 0\leq i < |s| \land_L res=\sum_{j=0}^{i-1}s[j]\}}$

          Y podemos ver que ambos $I\land B$ y la wp calculada son iguales, por lo tanto la tripla de Hoare es correcta

    \item $I \land \lnot B \implies Q_c$

          $0\leq i \leq |s| \land_L res=\sum_{j=0}^{i-1}s[j] \land \lnot (i<|s|)$

          $\equiv 0\leq i \leq |s| \land_L res=\sum_{j=0}^{i-1}s[j] \land i\geq|s|$

          $\equiv i = |s| \land_L res=\sum_{j=0}^{i-1}s[j]$

          $\equiv res=\sum_{j=0}^{|s|-1}s[j]$

          Que es efectivamente nuestra postcondicion de la especificacion.
\end{itemize}
\pagebreak

f)  Propuesta 1: $fv=|s|-i+1$

\salto{\baselineskip}

$\{I\land B \land v_0=fv\} res:=res+s[i];i:=i+1 \{fv<v_0\}\equiv Pc\implies wp(s,fv<v_0)$

\salto{\baselineskip}

$wp(S,fv<v_0)\equiv wp(res:=res+s[i], wp(i:=i+1, |s|-i+1<v_0))$

cosas...

\hspace*{2em}$\equiv |s|-(i+1)+1<v_0 \equiv |s|-i<v_0$

\salto{\baselineskip}

retomando el principio:

$I\land B \land v_0=fv \implies |s|-i<v_0$?

Si, pues $ v_o=fv \implies |s|-i<|s|-i+1$ que es trivialmente cierto.

Ahora, mi propuesta cumple con el segundo statement?

$I \land fv\leq 0 \implies \lnot B$

$\alpha:\ |s|-i+1\leq 0 \implies |s|+1\leq i$

$\beta:\ \lnot B \equiv i\geq |s| \equiv |s|\leq i\hspace*{5em} \longleftarrow $A lo que quiero llegar

$I\equiv 0\leq i \leq |s| \land_L res=\sum_{j=0}^{i-1}s[j]$

Tengo informacion de sobra, $\alpha \implies |s| \leq i \equiv \lnot B$

Seems a bit too easy pero es el primer ejercicio de la guia so...


\pagebreak


\includegraphics[width=\textwidth]{e2.png}

La precondicion del ciclo seria ...........................

La poscondicion es $res=\sum_{j=0}^{n-1}if\ j\ mod 2 = 0\ then\ j\ else\ 0$

$P\implies I$ es trivialmente cierto

\salto{\baselineskip}

$2^{da})$$\{I \land B \} res:=res+i; i:=i+2\{I\}$

    \salto{\baselineskip}

$I\equiv 0\leq i\leq n+1 \land i\%2=0 \land res=\sum_{j=0}^{i-1}if\ j\%2 = 0\ then\ j\ else\ 0$ {\small {\color{Violet}$\hfill \leftarrow$Por las dudas: Uso $\%$ en lugar de mod por cuestiones de spacing}}

$B\equiv i<n$

$wp(S,I)\equiv 0\leq i\leq n+1 \land i\%2=0 \land res+i=\sum_{j=0}^{i+1}if\ j\%2 = 0\ then\ j\ else\ 0$

    Analizando un poco nos damos cuenta de que i siempre sera par, por lo que i+1 es impar, entonces el termino i+1 de la sumatoria siempre sumara 0 (por el condicional), asi que podemos obviarlo. A su vez, podemos restar a ambos lados el termino i-esimo, que este sera par y por lo tanto si contara en la sumatoria, simplificando la expresion.

$wp(S,I)\equiv 0\leq i\leq n+1 \land i\%2=0 \land res=\sum_{j=0}^{i-1}if\ j\%2 = 0\ then\ j\ else\ 0$

    Entonces, $I\land B \equiv 0\leq i<n \land i\%2=0 \land res=\sum_{j=0}^{i-1}if\ j\%2 = 0\ then\ j\ else\ 0$

    Que efectivamente implica mi wp, pues el segundo termino es identico y en el primero,

$0\leq i<n\implies 0\leq i\leq n+1$

    \salto{\baselineskip}

$3^{ra})\ I\land\lnot B\implies Q_c$

$I\equiv 0\leq i\leq n+1 \land i\%2=0 \land res=\sum_{j=0}^{i-1}if\ j\%2 = 0\ then\ j\ else\ 0$

$\lnot B\equiv i\geq n$

$n\leq i \land i\leq n+1\equiv n\leq i\leq n+1$

    Pero i tiene que ser par, por lo que: $n\leq i\leq n+1\equiv n=i=n \equiv i=n$

$i=n \land i\%2=0 \land res=\sum_{j=0}^{i-1}if\ j\%2 = 0\ then\ j\ else\ 0 \equiv res=\sum_{j=0}^{n-1}if\ j\%2 = 0\ then\ j\ else\ 0 \equiv Q_c$

    Posible invariante: $n-i+1$

$\{I\land B\land v_0=fv\} S \{fv<v_0\}$

$wp(i:=i+2, n-i+1<v_0)\equiv n-i-1<v_0$

    entonces $v_0=fv \implies n-i-1<n-i+1$

$5^{ta})\ I\land fv\leq 0 \implies \lnot B$

    Quiero entonces llegar a $\lnot B \equiv i\geq n$

$fv\leq 0 \implies n+1\leq i$

$I\implies 0\leq i\leq n+1$

    Estas dos cosas implican que $i=n+1$ que a su vez implica $i\geq n$

    De forma un poco confusa y poco organizada, pero queda entonces demostrada la correctitud del ciclo.

    \pagebreak

    \includegraphics*[width=\textwidth]{./e3.png}

    while i<=n do

    if n mod i == 0 then

    res:=res+i

    i:=i+1

$P_c\equiv i=0 \land res=0$

$Q_c \equiv Q \equiv res=\sum_{j=1}^{n}\ if\ n\%j=0\ then\ j\ else\ 0$

    Propuesta de invariante:

$I\equiv 0\leq i\leq n+1 \land res=\sum_{j=1}^{i}\ if\ n\%j=0\ then\ j\ else\ 0$

    c) habria que cambiar la guarda del while, la pre y la post condicion no cambian.

    \pagebreak

    \includegraphics*[width=\textwidth]{e4.png}

    a) Todas?¿?¿ i, s y r

    b)

$I\equiv{\color{blue} 0\leq i \leq |s|}\land {\color{Red}|s|=|r|} \land \paraTodo[unalinea]{j}{\ent}{0\leq j < i \land r[j]=s[j]}$

    Lo rojo y azul es necesario para $P_c\implies I$

    Azul y negro para $\{B\land I\}S\{I\}$

    Denuevo, azul y negro para $I\land \lnot B \implies Q_c$

    c) fv=|s|-i

    d)

    \includegraphics*[width=0.5\textwidth]{solE4.png}

    Bueno no hacia falta lo rojo parece, fair enough

    \pagebreak

    \includegraphics*[width=\textwidth]{e5.png}

    Idealmente en papel haria un par de iteraciones del programa para identificar correctamente que es lo que hace, honestamente acá podria hacer una tabla y bla bla pero me da paja. Y ademas me parece que simplemente lo que hace es, de forma enrevesada, solo sumar la primera mitad de un arreglo de longitud par.

    Entonces mi propuesta de invariante es:

$I\equiv \frac{|s|}{2}-1\leq i \leq |s|-1 \land suma=\sum_{j=0}^{|s|-2-i}s[j]$

    Y mi propuesta de funcion variante es
$f_v=|s|/2-1-i$

    \includegraphics*[width=0.5\textwidth]{solE5.png}

    Ahora viendo la solucion me doy cuenta de que claro, como i esta "decreciendo" tengo que invertir los signos a como lo hago normalmente, porque asi como lo hice no se cumple que al finalizar el ciclo mi fv sea menor o igual a cero

    \pagebreak

    El seis me da paja, asi que va el siete.

    \includegraphics*[width=\textwidth]{e7.png}

    ----------------------------------------------------------------

    i:=0

    while $i<s.len()$ do

    \hspace*{3em}if $i\%2=0$ then

    \hspace*{6em}s[j]:=i*2

    \hspace*{3em}else

    \hspace*{6em}s[j]:=i*2+1

    ----------------------------------------------------------------

$P_c\equiv i=0$

$B\equiv i<s.len()$

$Q_c\equiv \paraTodo[unalinea]{i}{\ent}{0\leq i < |s| \land_L (i\%2=0 \land s[i]=2*i)\lor (i\%2\not=0 \land s[i]=i*2+1)}$

$fv=|s|-i$

    raro que no pida demostrar nada, quizas si me queda tiempo antes del recu intento hacerlo por los loles

    \pagebreak

    \includegraphics*[width=\textwidth]{e8.png}

    Aparentemente pone todo en cero desde el principio y el final no?

    Me parece raro que no tenga como requiere que la longitud de s sea par

    \rule{\textwidth}{0.5pt}

    i:=0

    while $i<|s|/2$

    \hspace*{3em}s[i]:=0

    \hspace*{3em}s[|s|-i-1]:=0

    \hspace*{3em}i:=i+1

    \rule{\textwidth}{0.5pt}

$P_c\equiv i=0 \land {\color{red}|s|\%2=0?}$

$B\equiv i<|s|/2$

$Q_c\equiv \paraTodo[unalinea]{i}{\ent}{0\leq i < |s| \land_L s[i]=0\land s[|s|-i-1]=0} $

    {\color{red}Puede ser que no haga falta lo de que sea par ya que de ultima si i y |s|-i-1 apuntan al mismo slot del array igualmente es valido porque ambos solo requieren que este sea igual a cero}

$fv=|s|/2 -i$

    \includegraphics*[width=\textwidth]{e9.png}
    {
        Estos dos predicados pareciera que hacen referencia a un i creciente, la duda que me surge es como determinar k (¿hace falta determinarlo?).

        Comparandolo con el predicado que usamos nosotros, que usa el signo invertido y k igual a cero, parece innecesariamente mas complejo, aunque imagino que es valido igualmente.

        Por ejemplo cuando recorres los indices de una lista, k seria |s| no? Es realmente tanto mas complejo entonces?

        Se me hace que k siempre seria esa primera parte que le pongo al invariante, a la que basicamente le resto i de alguna forma mas o menos compleja.

        Esto haria entonces que fv sea simplemente i (y sus cosas extra), y de hecho esto tiene sentido porque

        $fv=i\land k=|s| \implies fv\geq k \equiv i\geq |s| \equiv i-|s|\geq 0 \equiv |s|-i<0$

        Not quite there :/

        tl;dr: ni idea, creo que si pero no se como fundamentar mi respuesta, parece una generalizacion del predicado que usamos.
    }

    \pagebreak

    \includegraphics*[width=\textwidth]{e10.png}

    Habria que comenzar viendo cual es la Q principal, osea la de la especificacion, a la que llamare $Q_e$

$Q_e\equiv res=true \sii \existe[unalinea]{k}{\ent}{0\leq k < |s| \land_L s[k]=e}$

    Ahora de forma secuencial puedo comenzar a calcular las wp por linea/instruccion

    \begin{align*}
        wp(\text{if $j!=1$ then $res:=true$ else $res:=false$}, Q_e)\equiv & def(b)\land_L (j\not= -1\land Q_{true}^{res}) \lor (j=-1 \land Q_{false}^{res})            \\
        \equiv                                                             & (j\not=-1 \land true=true \sii \existe[unalinea]{k}{\ent}{0\leq k < |s| \land_L s[k]=e})   \\
                                                                           & \lor (j=-1 \land false=true \sii \existe[unalinea]{k}{\ent}{0\leq k < |s| \land_L s[k]=e}) \\
        \# 1\equiv                                                         & (j\not=-1 \land \existe[unalinea]{k}{\ent}{0\leq k < |s| \land_L s[k]=e})                  \\
                                                                           & \lor (j=-1 \land \paraTodo[unalinea]{k}{\ent}{0\leq k < |s| \land_L s[k]\not=e}
        )
    \end{align*}
    \#1) Acá es confuso, dado que \emph{q sii p} es una expresion que es verdadera si q y p son falsos o verdaderos simultaneamente, creo que puedo reducirlo a lo escrito dado que
    \begin{multline*}
        \\q \sii p \equiv q\implies p \land p \implies q
        \\\text{si }q=true
        \\true \implies p \land p \implies true
        \\p \land p
        \\p
        \\\text{si }q=false
        \\false \implies p \land p \implies false
        \\true \land \lnot p
        \\\lnot p
        \\
    \end{multline*}

    Voy a llamar a esta postcondicion $Q_i$ por ser la wp de mi $Q_e$ y el if exterior.

$Q_i\equiv (j\not=-1 \land \existe[unalinea]{k}{\ent}{0\leq k < |s| \land_L s[k]=e})
\lor (j=-1 \land \paraTodo[unalinea]{k}{\ent}{0\leq k < |s| \land_L s[k]\not=e})$

    Justo tambien es la postcondicion de mi ciclo, asi que ahora tengo que demostrar la correctitud del ciclo

$P_c\equiv i=0 \land j=-1$

$B_c\equiv i<|s|$

$S\equiv \text{if s[i]=e then j:=i else skip};i:=i+1$

    Propuesta de invariante:

$I\equiv 0\leq i \leq |s| \land -1\leq j< |s|$

    ¿$P_c\implies I$?

$i=0 \land j=-1 \implies I\equiv 0\leq i \leq |s| \land -1\leq j< |s| \hfill \blacksquare$

\vbox{

¿$\{I\land B\}S\{I\}$?

Esto equivale a ver si $I\land B \implies wp(S,I)$, asi que comienzo calculando dicha wp

$wp(i:=i+1, I) \equiv 0\leq i+1 \leq |s| \land -1\leq j< |s|$

$wp(i:=i+1, I) \equiv -1 \leq i \leq |s|-1 \land -1\leq j< |s|$

Ahora con el if:

$wp(\text{if s[i]=e then  j:=i else skip},wp(i:=i+1, I))\equiv def(B) \land ((B \land Q_i^j) \lor (\lnot B \land ¿true?))$

$0\leq i < |s| \land_L ((s[i]=e \land -1 \leq i \leq |s|-1 \land -1\leq i < |s|)\lor s[i]\not=e)$

$0\leq i < |s| \land_L (s[i]=e \lor s[i]\not=e)$

$0\leq i < |s|$

Se siente raro

$I\land B \implies 0\leq i < |s| \hfill \blacksquare?$

¿$I\land \lnot B \implies Q_i$?

Esto es medio raro de demostrar, no se si se puede de hecho. 
Siento que I y no B no me dan suficiente informacion sobre la postcondicion,
o es que estoy en una situacion de algo pseudo true implies whatever?
estoy en una situacion de true implies p o no p? parece no?
Asumo entonces que estoy en lo correcto y que la desmotracion es valida (?)

Una posible funcion variante es: $fv=|s|-i$

$\{I\land B\land fv=v_0\}S{fv<v_0}$

Trivialmente cierto pues i crece en cada iteracion.

¿$I\land fv\leq 0 \implies \lnot B$?

$I\land fv\leq 0 \equiv 0\leq i \leq |s| \land |s|-i\leq 0 $

$0\leq i \leq |s| \land |s|-1\leq i \equiv |s|\leq i \leq |s| \implies i\geq|s| \equiv \lnot B\hfill\blacksquare$

Muy dudoso todo pero tengo sueño y hambre y es tarde y no quiero saber mas nada de esto. Mañana sera otro dia.

}

\end{document}