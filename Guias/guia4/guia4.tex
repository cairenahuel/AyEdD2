\documentclass[10pt,a4paper]{article}
\usepackage{tabularx}
\usepackage{amssymb}
\usepackage{enumitem}
\usepackage[spanish,activeacute,es-tabla]{babel}
\usepackage[utf8]{inputenc}
\usepackage{ifthen}
\usepackage{listings}
\usepackage{dsfont}
\usepackage{subcaption}
\usepackage{amsmath}
\usepackage[top=1cm,bottom=2cm,left=1cm,right=1cm]{geometry}%
\usepackage{color}%
\usepackage{changepage}
\newcommand{\tocarEspacios}{%
	\addtolength{\leftskip}{3em}%
	\setlength{\parindent}{0em}%
}

% Especificacion de procs

\newcommand{\In}{\textsf{in }}
\newcommand{\Out}{\textsf{out }}
\newcommand{\Inout}{\textsf{inout }}

\newcommand{\encabezadoDeProc}[4]{%
	% Ponemos la palabrita problema en tt
	%  \noindent%
	{\normalfont\bfseries\ttfamily proc}%
	% Ponemos el nombre del problema
	\ %
	{\normalfont\ttfamily #2}%
	\
	% Ponemos los parametros
	(#3)%
	\ifthenelse{\equal{#4}{}}{}{%
		% Por ultimo, va el tipo del resultado
		\ : #4}
}

\newenvironment{proc}[4][res]{%
	
	% El parametro 1 (opcional) es el nombre del resultado
	% El parametro 2 es el nombre del problema
	% El parametro 3 son los parametros
	% El parametro 4 es el tipo del resultado
	% Preambulo del ambiente problema
	% Tenemos que definir los comandos requiere, asegura, modifica y aux
	\newcommand{\requiere}[2][]{%
		{\normalfont\bfseries\ttfamily requiere}%
		\ifthenelse{\equal{##1}{}}{}{\ {\normalfont\ttfamily ##1} :}\ %
		\{\ensuremath{##2}\}%
		{\normalfont\bfseries\,\par}%
	}
	\newcommand{\asegura}[2][]{%
		{\normalfont\bfseries\ttfamily asegura}%
		\ifthenelse{\equal{##1}{}}{}{\ {\normalfont\ttfamily ##1} :}\
		\{\ensuremath{##2}\}%
		{\normalfont\bfseries\,\par}%
	}
	\renewcommand{\aux}[4]{%
		{\normalfont\bfseries\ttfamily aux\ }%
		{\normalfont\ttfamily ##1}%
		\ifthenelse{\equal{##2}{}}{}{\ (##2)}\ : ##3\, = \ensuremath{##4}%
		{\normalfont\bfseries\,;\par}%
	}
	\renewcommand{\pred}[3]{%
		{\normalfont\bfseries\ttfamily pred }%
		{\normalfont\ttfamily ##1}%
		\ifthenelse{\equal{##2}{}}{}{\ (##2) }%
		\{%
		\begin{adjustwidth}{+5em}{}
			\ensuremath{##3}
		\end{adjustwidth}
		\}%
		{\normalfont\bfseries\,\par}%
	}
	
	\newcommand{\res}{#1}
	\vspace{1ex}
	\noindent
	\encabezadoDeProc{#1}{#2}{#3}{#4}
	% Abrimos la llave
	\par%
	\tocarEspacios
}
{
	% Cerramos la llave
	\vspace{1ex}
}

\newcommand{\aux}[4]{%
	{\normalfont\bfseries\ttfamily\noindent aux\ }%
	{\normalfont\ttfamily #1}%
	\ifthenelse{\equal{#2}{}}{}{\ (#2)}\ : #3\, = \ensuremath{#4}%
	{\normalfont\bfseries\,;\par}%
}

\newcommand{\pred}[3]{%
	{\normalfont\bfseries\ttfamily\noindent pred }%
	{\normalfont\ttfamily #1}%
	\ifthenelse{\equal{#2}{}}{}{\ (#2) }%
	\{%
	\begin{adjustwidth}{+2em}{}
		\ensuremath{#3}
	\end{adjustwidth}
	\}%
	{\normalfont\bfseries\,\par}%
}

% Tipos

\newcommand{\nat}{\ensuremath{\mathds{N}}}
\newcommand{\ent}{\ensuremath{\mathds{Z}}}
\newcommand{\float}{\ensuremath{\mathds{R}}}
\newcommand{\bool}{\ensuremath{\mathsf{Bool}}}
\newcommand{\cha}{\ensuremath{\mathsf{Char}}}
\newcommand{\str}{\ensuremath{\mathsf{String}}}

% Logica

\newcommand{\True}{\ensuremath{\mathrm{true}}}
\newcommand{\False}{\ensuremath{\mathrm{false}}}
\newcommand{\Then}{\ensuremath{\rightarrow}}
\newcommand{\Iff}{\ensuremath{\leftrightarrow}}
\newcommand{\implica}{\ensuremath{\longrightarrow}}
\newcommand{\IfThenElse}[3]{\ensuremath{\mathsf{if}\ #1\ \mathsf{then}\ #2\ \mathsf{else}\ #3\ \mathsf{fi}}}
\newcommand{\yLuego}{\land _L}
\newcommand{\oLuego}{\lor _L}
\newcommand{\implicaLuego}{\implica _L}

\newcommand{\cuantificador}[5]{%
	\ensuremath{(#2 #3: #4)\ (%
		\ifthenelse{\equal{#1}{unalinea}}{
			#5
		}{
			$ % exiting math mode
			\begin{adjustwidth}{+2em}{}
				$#5$%
			\end{adjustwidth}%
			$ % entering math mode
		}
		)}
}

\newcommand{\existe}[4][]{%
	\cuantificador{#1}{\exists}{#2}{#3}{#4}
}
\newcommand{\paraTodo}[4][]{%
	\cuantificador{#1}{\forall}{#2}{#3}{#4}
}

%listas

\newcommand{\TLista}[1]{\ensuremath{seq \langle #1\rangle}}
\newcommand{\lvacia}{\ensuremath{[\ ]}}
\newcommand{\lv}{\ensuremath{[\ ]}}
\newcommand{\longitud}[1]{\ensuremath{|#1|}}
\newcommand{\cons}[1]{\ensuremath{\mathsf{addFirst}}(#1)}
\newcommand{\indice}[1]{\ensuremath{\mathsf{indice}}(#1)}
\newcommand{\conc}[1]{\ensuremath{\mathsf{concat}}(#1)}
\newcommand{\cab}[1]{\ensuremath{\mathsf{head}}(#1)}
\newcommand{\cola}[1]{\ensuremath{\mathsf{tail}}(#1)}
\newcommand{\sub}[1]{\ensuremath{\mathsf{subseq}}(#1)}
\newcommand{\en}[1]{\ensuremath{\mathsf{en}}(#1)}
\newcommand{\cuenta}[2]{\mathsf{cuenta}\ensuremath{(#1, #2)}}
\newcommand{\suma}[1]{\mathsf{suma}(#1)}
\newcommand{\twodots}{\ensuremath{\mathrm{..}}}
\newcommand{\masmas}{\ensuremath{++}}
\newcommand{\matriz}[1]{\TLista{\TLista{#1}}}
\newcommand{\seqchar}{\TLista{\cha}}

\renewcommand{\lstlistingname}{Código}
\lstset{% general command to set parameter(s)
	language=Java,
	morekeywords={endif, endwhile, skip},
	basewidth={0.47em,0.40em},
	columns=fixed, fontadjust, resetmargins, xrightmargin=5pt, xleftmargin=15pt,
	flexiblecolumns=false, tabsize=4, breaklines, breakatwhitespace=false, extendedchars=true,
	numbers=left, numberstyle=\tiny, stepnumber=1, numbersep=9pt,
	frame=l, framesep=3pt,
	captionpos=b,
}
\usepackage{changepage}
\usepackage{xcolor}
\usepackage[T1]{fontenc}
\usepackage{listings}
\usepackage{color}
\usepackage{amssymb}

\definecolor{dkgreen}{rgb}{0,0.6,0}
\definecolor{gray}{rgb}{0.5,0.5,0.5}
\definecolor{mauve}{rgb}{0.58,0,0.82}

\lstset{frame=tb,
  language=Java,
  aboveskip=3mm,
  belowskip=3mm,
  showstringspaces=false,
  columns=flexible,
  basicstyle={\small\ttfamily},
  numbers=none,
  numberstyle=\tiny\color{gray},
  keywordstyle=\color{blue},
  commentstyle=\color{dkgreen},
  stringstyle=\color{mauve},
  breaklines=true,
  breakatwhitespace=true,
  tabsize=3
}
\newcommand{\noexiste}[4][]{%
	\cuantificador{#1}{\nexists}{#2}{#3}{#4}
}
\renewcommand*\ttdefault{txtt}
\renewcommand*\familydefault{\ttdefault} %% Only if the base font of the document is to be typewriter style
\newcommand{\tbft}[2]{\par\addvspace{\baselineskip}\textbf{#1}\hspace{0.35em}{#2}\\\par\addvspace{\baselineskip}}
\newcommand{\ejercicio}[2]{\par\addvspace{\baselineskip}\textbf{Ejercicio #1.}\hspace{0.35em}{#2}\\\par\addvspace{\baselineskip}}
% \ejercicio{NUMERO}{ENUNCIADO}
%   Devuelve
% Ejercicio NUMERO. ---- ENUNCIADO ----
%
%formato
\newcommand{\salto}[1]{\par\addvspace{#1}}
\newcommand{\rojo}[1]{{\color{red}#1}}
\newcommand{\anotacion}[2][red]{\salto{1ex}\noindent\texttt{\color{#1}#2}\salto{1ex}}
\newcommand{\anotacionns}[2][red]{\noindent\texttt{\color{#1}#2}}
\newcommand{\return}{\textbf{return }}
%
% si y solo si corto y largo
\newcommand{\sii}{\leftrightarrow}
\newcommand{\siiLargo}{\longleftrightarrow}
\newcommand{\slr}[1]{\ensuremath{\langle #1\rangle}}
\newcommand{\smm}[1]{\textless #1\textgreater}
\newcommand{\encabezadoTAD}[1]{\par\salto{1ex}\noindent TAD\ \ \normalfont\ttfamily#1 }
\newenvironment{tad}[1]{
\newcommand{\nombretad}{{\ttfamily#1}}
\newcommand{\nt}{\nombretad}
\newcommand{\obs}[2]{\par\noindent{\ttfamily obs} ##1: ##2\par}
\encabezadoTAD{#1}\{
    \begin{adjustwidth}{3em}{0em}}
{\end{adjustwidth}\par\}}
% \begin{tad}{nombre del tad}
%   AGREGAR UN OBSERVADOR
%     \obs{nombre observador}{tipo}
%     \nombretad <---------- DEVUELVE EL NOMBRE DEL TAD (du)
%
%   y aca se pueden usar todos los procs y cosas de la catedra
%
%     \end{tad}
\newcommand{\encabezadoImpl}[3]{\par\salto{1ex}\noindent \textbf{impl}\ \normalfont\ttfamily#1(#2):#3}
\newcommand{\asg}[2]{\salto{0em}\noindent{\ttfamily #1}:= #2\salto{0em}}
\newcommand{\asgns}[2]{\noindent{\ttfamily #1}:= #2}
\newenvironment{impl}[3]{\encabezadoImpl{#1}{#2}{#3}\{
\begin{adjustwidth}{3em}{0em}
}
{\end{adjustwidth}\par\}}
\newcommand{\encabezadoDesign}[2]{\par\salto{1ex}\noindent \textbf{modulo}\ {\normalfont\ttfamily #1} \textbf{implementa}{ \normalfont\ttfamily #2}}
\newenvironment{design}[2]{\encabezadoDesign{#1}{#2}\{
\newcommand{\var}[2]{\salto{0em}\noindent{\normalfont \bfseries var} \texttt{##1}: \texttt{##2}\salto{0em}}
\begin{adjustwidth}{3em}{0em}
}
{\end{adjustwidth}\}}
\newcommand{\ifthel}[3]{\salto{0ex}
\noindent{\normalfont \bfseries if}{ #1 }{\normalfont \bfseries then}\salto{0ex}
\noindent{\begin{adjustwidth}{1.5em}{0em}#2\end{adjustwidth}}\salto{0ex}
\noindent{\normalfont \bfseries else}\begin{adjustwidth}{1.5em}{0em}{#3}\end{adjustwidth}\salto{0ex}
\noindent{\bfseries end if}}\salto{0ex}
\newcommand{\while}[2]{
    \salto{0em}\noindent
    {\normalfont \bfseries while} \ensuremath{#1} \textbf{do}
    \begin{adjustwidth}{1.5em}{0em}{#2}\end{adjustwidth}\salto{0em}\noindent{\normalfont \bfseries end while}
}
\newcommand{\invrep}[2]{\salto{0em}\noindent{\bfseries pred} InvRep (#1)\{\salto{0em}\noindent\begin{adjustwidth}{2em}{0em}{#2}\end{adjustwidth}\salto{0em}\noindent\}}
\newcommand{\abs}[2]{\salto{0em}\noindent{\bfseries pred} Abs (#1)\{\salto{0em}\noindent\begin{adjustwidth}{2em}{0em}{#2}\end{adjustwidth}\salto{0em}\noindent\}}

\begin{document}
\section*{Especificacion de TADs}
\tbft{Ejercicio 1}{Especificar en forma completa el TAD \texttt{NumeroRacional} que incluya al menos las operaciones aritmeticas basicas}
\begin{tad}{NumeroRacional}
    \obs{n}{\ent}
    \obs{d}{\ent}
    \begin{proc}{suma}{\In a,b: \nombretad}{\nombretad}
        \requiere{true}
        \asegura{res=\frac{a.n*b.d+a.d*b.n}{b.d*a.d}}
    \end{proc}
    \begin{proc}{resta}{\In a,b: \nombretad}{\nombretad}
        \requiere{true}
        \asegura{res=\frac{a.n*b.d-a.d*b.n}{b.d*a.d}}
    \end{proc}
    \begin{proc}{multiplicacion}{\In a,b: \nombretad}{\nombretad}
        \requiere{true}
        \asegura{res=\frac{a.n*b.n}{b.d*a.d}}
    \end{proc}
    \begin{proc}{division}{\In a,b: \nombretad}{\nombretad}
        \requiere{b\not= 0}
        \asegura{res=\frac{a.n*b.d}{b.n*a.d}}
    \end{proc}
\end{tad}
\tbft{Ejercicio 2:}{Especificar TADs para las siguientes figuras geometricas. Tiene que contener las operaciones rotar, trasladar y escalar, y una mas propuesta por usted.}
\begin{itemize}
    \item Rectangulo (2D)
    \item Esfera (3D)
\end{itemize}
\begin{tad}{Rectangulo}
    \obs{alto}{\float}
    \obs{ancho}{\float}
    \obs{posicion}{$\langle\float\times\float\rangle$}
    \obs{angulo}{\float}
    \begin{proc}{nuevo\nombretad}{\In h: \float, \In w: float, \In x:\float,\In y:\float}{\nombretad}
        \requiere{w \geq 0 \land h \geq 0}
        \asegura{res.ancho=w \land res.alto=h \land res.posicion=(x,y) \land r.angulo=0}
    \end{proc}
    \begin{proc}{escalar}{\Inout r: \nombretad, mh:\float, mw:\float}{}
        \requiere{r=R_0}
        \asegura{(mw\geq0 \land mh \geq0) \yLuego r.posicion=R_0.posicion }
        \asegura{r.alto=|R_0.alto*mh| \land r.ancho=|R_0.ancho*mw|}
        \asegura{mw<0 \yLuego r.posicion_1=R_0.posicion_1+R_0.ancho*mw}
        \asegura{mh<0 \yLuego r.posicion_2=R_0.posicion_2+R_0.alto*mh}
    \end{proc}
    \begin{proc}{rotar}{\Inout r: \nombretad, \In rad: \float}{}
        \requiere{r=R_0}
        \asegura{r.angulo=R_0.angulo+rad}
    \end{proc}
    \begin{proc}{trasladar}{\Inout r: \nombretad, \In pos: $\langle x,y\rangle$}{}
        \requiere{r=R_0}
        \asegura{r.posicion=R_0.posicion+pos}
    \end{proc}
    \begin{proc}{area}{\In r: \nombretad}{\float}
        \asegura{res=r.alto \cdot r.ancho}
    \end{proc}
\end{tad}
\pagebreak
\begin{tad}{Esfera}
    \obs{radio}{\float}
    \obs{centro}{$\langle \float\times\float\times\float \rangle$}
    %\vspace*{0.5em}
    %\noindent Por claridad de notacion $centro_x$, $centro_y$ y $centro_z$ seran\\usadas en lugar de sus variantes con numero.
    \obs{semieje}{$\langle \float\times\float\times\float \rangle$}
    \begin{proc}{nueva\nombretad}{\In r: \float, \In pos:$\langle \float\times\float\times\float \rangle$}{\nombretad}
        \asegura{res.radio={|r|}}
        \asegura{res.centro=pos}
    \end{proc}
    \begin{proc}{escalar}{\Inout esfera: \nombretad, \In e: \float}{}
        \requiere{esfera=esfera_0}
        \asegura{esfera.radio=ESFERA_0.radio*|e|}
    \end{proc}
    \begin{proc}{trasladar}{\Inout esfera: \nombretad, \In pos:$\langle x,y,z\rangle$}{}
        \requiere{esfera=ESFERA_0}
        \asegura{esfera.centro=ESFERA_0.centro+pos}
    \end{proc}
    \begin{proc}{rotar}{\Inout esfera: \nombretad, \In angulosrad: $\langle \alpha,\beta,\gamma \rangle$}{}
        \requiere{esfera=ESFERA_0}
        \asegura{esfera.semieje=ESFERA_0.semieje+angulosrad}
    \end{proc}
\end{tad}
\ejercicio{3}{Especifique el TAD DobleCola\slr{T}, en el que los elementos pueden insertarse al principio o al final y se eliminan por el medio.}
\begin{tad}{DobleCola}
    \obs{elems}{\TLista{T}}
    \begin{proc}{\nombretad Vacia}{}{\nt}
        \asegura{res=\slr{}}
    \end{proc}
    \begin{proc}{encolarFinal}{\Inout doblecola: \nt, \In e: T}{}
        \requiere{doblecola=DOBLECOLA_0}
        \asegura{doblecola.elems=concat(DOBLECOLA_0.elems, \langle e\rangle)}
    \end{proc}
    \begin{proc}{encolarInicio}{\Inout doblecola: \nt, \In e: T}{}
        \requiere{doblecola=DOBLECOLA_0}
        \asegura{doblecola.elems=concat(\langle e\rangle,DOBLECOLA_0.elems)}
    \end{proc}
    \begin{proc}{desencolar}{\Inout doblecola: \nt}{T}
        \requiere{doblecola=DOBLECOLA_0}
        \requiere{cola.elems\not=\slr{}}
        \asegura{res\not \in doblecola.elems}
        \asegura{res=DOBLECOLA_0.elems\Biggl[\Bigl\lceil{\frac{|DOBLECOLA_0|}{2}}\Bigr\rceil\Biggr]}
    \end{proc}
\end{tad}
\pagebreak
\ejercicio{4}{Especifique el TAD DiccionarioConHistoria. El mismo guarda para cada clave, todos los valores que se asociaron con la misma a lo largo del tiempo (en orden)}
\begin{tad}{DiccionarioConHistoria\<T,\TLista{K}\>}
    \obs{data}{dict<T,\TLista{K}>}
    \begin{proc}{\nt Vacio}{}{\nt}
        \asegura{res.data=\{\}}
    \end{proc}
    \begin{proc}{estaLaLlave}{\In dic:\nt, \In e: T}{\bool}
        \asegura{res=true \siiLargo e \in dic.data}
    \end{proc}
    \begin{proc}{definir}{\Inout dic: \nt,\In k: T, \In e: K}{}
        \requiere{DIC_0=dic \yLuego k \in RES_0.data}
        \asegura{dic.data[k][0]=setKey(DIC_0.data,k,concat(DIC_0.data[k],e))}
    \end{proc}
    \begin{proc}{consultarHistorial}{\In dic: \nt, \In k: T}{\TLista{K}}
        \asegura{res=res.data[k]}
    \end{proc}
    \begin{proc}{borrar}{\Inout dic \nt,\In k: T}{}
        \requiere{dic=DIC_0}
        \requiere{k\in dic.data}
        \asegura{dic.data=delKey(DIC_0, k)}
    \end{proc}
    \begin{proc}{tamaño}{\In dic: \nt}{\ent}
        \asegura{res=|dic.data|}
    \end{proc}
\end{tad}
\ejercicio{5}{Modifique el TAD ColaDePrioridad<T> para que, si hay muchos valores iguales al maximo, la operacion desapilarMax los desapile a todos.}
\begin{tad}{ColaDePrioridad\textless T\textgreater}
    ---Toda la implementacion normal de \nt ---
    \begin{proc}{desapilarMax}{\Inout cola:\nt}{\TLista{T}}
        \requiere{cola=COLA_0}
        \asegura{\not \exists\ e \in cola.data, \exists e' \in COLA_0.data: tienePriMax(COLA_0.data, e') \yLuego cola.data[e] = COLA_0.data[e']}
        \asegura{(\forall e \in COLA_0.data)\ (tienePriMax(COLA_0.data, e) \yLuego e \in res)\lor\\ (\lnot tienePriMax(COLA_0.data, e) \yLuego e\not\in res)}
        {\color{red}\asegura{|res|=|COLA_0.data|-|cola.data|}}
    \end{proc}
\end{tad}
\\\texttt{\color{red}No estoy seguro de si se puede usar el operando $\in$ con diccionarios, ni si estoy asegurando que TODOS los 
elementos que saco de la cola sean incluidos en la sequencia de resultado, por ejemplo si tengo dos elementos en la cola con el 
mismo nombre y la misma prioridad, ambos van a la secuencia con la definicion que di? creo que no, que como uno ya pertenece el otro 
no necesariamente se agrega. ¿Comparar la cantidad que saco y el largo de la secuencia puede ser una opcion valida?}
\ejercicio{6}{Especifique los TADS indicados a continuacion pero utilizando los observadores propuestos}
\begin{itemize}
    \item Diccionario\smm{K,V} observando con conjuntos (de tuplas)
    \item Conjunto\smm{T} observando con funciones
    \item Pila\smm{T} observando con diccionarios
    \item Punto observando con coordenadas polares
\end{itemize}
\pagebreak
\begin{tad}{Diccionario\smm{K,V}}
    \obs{datos}{conj\slr{K\times V}}
    \begin{proc}{diccionarioVacio}{}{\nt}
        \asegura{res.datos=\{\}}
    \end{proc}
    \begin{proc}{agregar}{\Inout dic: \nt, \In k: K, \In v: V}{}
        \requiere{dic=DIC_0}
        \asegura{(k,v) \in dic.datos}
    \end{proc}
    {\color{red}---Se puede hacer siquiera esto? Otro observador con otro conjunto de tuplas
     y apariciones? raro}
\end{tad}
\salto{\baselineskip}
\begin{tad}{Conjunto\smm{T}}
    \obs{pertenece}{\bool}
    \obs{largo}{\ent}
    \begin{proc}{conjVacio}{}{\nt}
        \requiere{true}
        \asegura{res.largo=0}
    \end{proc}
    \begin{proc}{agregar}{\Inout c: \nt, \In e: T}{}
        \requiere{c=C_0}
        \asegura{c.pertenece(e)}
    \end{proc}
    \begin{proc}{sacar}{\Inout c: \nt, \In e: T}{}
        \requiere{c=C_0}
        \asegura{\lnot c.pertenece(e)}
    \end{proc} 
    \begin{proc}{unir}{\Inout c: \nt, \In c': \nt}{}
        \requiere{c=C_0}
        \asegura{\forall e \in T, c'.pertenece(e): c.pertenece(e)}
    \end{proc}
    \begin{proc}{restar}{\Inout c: \nt, \In c': \nt}{}
        \requiere{c=C_0}
        \asegura{\forall e \in T, c'.pertenece(e): \lnot c.pertenece(e)}
    \end{proc}
    \begin{proc}{intersecar}{\Inout c: \nt, \In c': \nt}{}
        \requiere{c=C_0}
        \asegura{\forall e \in T, C_0.pertenece(e) \land c'.pertenece(e) \sii c.pertenece(e)}
    \end{proc}
    \begin{proc}{tamaño}{\In c: \nt}{\nat}
        \requiere{true}
        \asegura{\exists! m \in \ent: \sum_{i=1}^{m}\IfThenElse{c.pertenece(e)}{1}{0}=res, e\in T}
        {\color{red} --- acá cree el observador largo porque muchas cosas raras tenia que hacer}
        \\\asegura{res=c.largo}
    \end{proc}
\end{tad}
\salto{\baselineskip}
\begin{tad}{Pila\smm{T}}
    \obs{elems}{dict\smm{\nat,T}}
    \begin{proc}{pilaVacia}{}{\nt}
        \requiere{true}
        \asegura{res.elems=\{\}}
    \end{proc}
    \begin{proc}{vacia}{\In pila: \nt}{\bool}
        \requiere{true}
        \asegura{res=true\sii pila.elems=\{\}}
    \end{proc}
    \begin{proc}{apilar}{\Inout pila: \nt, \In e: T}{}
        \requiere{pila=PILA_0}
        \asegura{|PILA_0.elems|+1=|pila.elems| \land |pila.elems|\in pila.elems \yLuego setKey(pila.elems, |pila.elems|, e)}
    \end{proc}
    \pagebreak
    \begin{proc}{desapilar}{\Inout pila: \nt}{T}
        \requiere{pila=PILA_0}
        \requiere{pila.elems\not=\{\}}
        \asegura{|pila.elems|=|PILA_0.elems|-1\land |PILA_0.elems|\not\in pila.elems}
        \asegura{res=PILA_0.elems\big[|PILA_0.elems|\big]}
    \end{proc}
    \begin{proc}{tope}{\In pila: \nt}{T}
        \requiere{pila.elems\not=\{\}}
        \asegura{res=pila.elems\big[|pila.elems|-1\big]}
    \end{proc}
\end{tad}
\salto{\baselineskip}
\begin{tad}{Punto}
    \obs{coords}{\slr{\float \times \float}}
    \pred{igualdad}{p1, p2: \nt}{(\exists k \in \ent: p1.coords_2=p2.coords_2+2k\pi)\land(p1.coords_1=p2.coords_1)}
    \begin{proc}{crearPunto}{\In r: \float, \In $\alpha$: \float}{\nt}
        \requiere{r\geq0}
        \asegura{res.coords=(r,\alpha)}
    \end{proc}
    \begin{proc}{mover}{\Inout p: \nt, \In newcoords: \slr{\float\times\float}}{}
        \requiere{true}
        \asegura{p.coords_1=newcoords_1\land p.coords_2=newcoords_2}
    \end{proc}
\end{tad}
\ejercicio{7}{Especificar TADs para las siguientes estructuras:}
\begin{itemize}
    \item Multiconjunto\smm{T} -  Es igual a un conjunto pero con duplicados. Cada elemento tiene asociada una multiplicidad,
     que es la cantidad de veces que este aparece en la estrucutra. Tiene las mismas operaciones que un conjunto y ademas una que 
     indica la multiplicidad del elemento.
     \begin{tad}{Multiconjunto}
         \obs{elems}{dict\smm{T,\nat}}
         \begin{proc}{conjuntoVacio}{}{\nt}
             \requiere{true}
             \asegura{res.elems=\{\}}
         \end{proc}
         \begin{proc}{agregar}{\Inout mc: \nt, \In e: T}{}
             \requiere{mc=MC_0}
             \asegura{e\in MC_0.elems \yLuego mc.elems[e]=MC_0.elems[e]+1}
             \asegura{e\not\in MC_0.elems \yLuego setKey(mc,e,1)}
         \end{proc}
         \begin{proc}{eliminar}{\Inout mc: \nt}{}
             \requiere{mc=MC_0}
             \asegura{e\in MC_0.elems \yLuego MC_0.elems[e]=1 \yLuego mc.elems=delKey(MC_0,e)}
             \asegura{e\in M_0.elems \yLuego MC_0.elems[e]>1 \yLuego mc.elems=setKey(MC_0,e,MC_0.elems[e]-1)}
         \end{proc}
         \begin{proc}{multiplicidad}{\In mc: \nt, \In e: T}{\ent}
             \requiere{true}
             \asegura{e \in mc \yLuego res=mc.elems[e]}
             \asegura{e\not\in mc \yLuego res=0}
         \end{proc}
         \rojo{--- Hay mas operaciones pero creo que estas son las mas relevantes}
     \end{tad}
     \pagebreak
     \item Multidict\smm{K,V}: Misma idea pero para diccionarios, cada clave puede estar asociada a multiples valores. Los valores se definen de a uno, 
     pero la operacion obtener debe devolver todos los valores asociados a una determinada clave.\\ 
     \rojo{\\Sobre la nota(que no copie): una posible implementacion que se me ocurre es la de un taller en el que las keys son los operarios y 
     los values son listas en las que estan los trabajos pendientes que le corresponden a cada uno, en una implementacion mas completa se 
     podria hacer procs para mover los trabajos de la cola (porque la secuencia va a ser basicamente una cola en su comportamiento default) 
     pero que tambien se pueda elegir trabajos especificos que hacer para darles prioridad.}
     \begin{tad}{Multidict\smm{K,V}}
         \obs{elems}{dict\smm{K,\TLista{V}}}
         \begin{proc}{multidictVacio}{}{\nt}
             \requiere{true}
             \asegura{res.elems=\{\}}
         \end{proc}
         \begin{proc}{agregar}{\Inout md: \nt, \In k: K, \In v: V}{}
             \requiere{md=MD_0}
             \asegura{k\in MD_0.elems \yLuego setKey(md.elems, k, concat(\slr{v},MD_0.elems[k]))}
             \asegura{k\not\in MD_0.elems \yLuego setKey(md.elems, k, \slr{v})}
         \end{proc}
         \begin{proc}{borrar}{\Inout md: \nt, \In k: K}{}
             \requiere{md=MD_0}
             \asegura{k\not\in MD_0 \yLuego md=MD_0}
             \asegura{k\in MD_0.elems \land |MD_0.elems|=1 \yLuego delKey(md.elems, k)}
             \asegura{k\in MD_0.elems \land |MD_0.elems|>0 \yLuego\\\hspace*{4.5em}setKey(md.elems, k, subseq(MD_0.elems, 1, |MD_0.elems|[k]))}
         \end{proc}
         \rojo{--- Creo que el resto de la implementacion es trivial, podria agregarse un booleando como parametro de borrar que determine si
          se borra la key completamente o si se borra valor a valor}
     \end{tad}
    \end{itemize}
\ejercicio{8}{Especifique el TAD contadores que, dada una lista de eventos, permite contar la cantidad de veces que se produjo cada uno de ellos.
La lista de eventos es fija. El TAD debe tener una operacion para incrementar el contador asociado a un evento y una operacion para conocer 
el valor actual del contador de dicho evento.\\
\hspace*{2em}-Modifique el TAD para que sea posible preguntar el valor del contador en un determinado momento del pasado. Si necesita conocer la fecha 
y hora actual puede pasarla como parametro a los procedimientos. Asuma que las dechas son numeros enteros (por ejemplo la cantidad de segundos desde el 1ro de enero del '70)}
\begin{tad}{Contadores}
    \obs{elems}{dict\smm{T,\ent}}
    \begin{proc}{nuevosContadores}{\In l: \TLista{T}}{\nt}
        \requiere{true}
        \asegura{|c.elems|=|l||}
        \asegura{\forall e \in l: e\in res.elems \yLuego res[e]=0}
    \end{proc}
    \begin{proc}{aumentarContador}{\Inout c: \nt, \In k: T}{}
        \requiere{c=C_0}
        \requiere{k\in c.elems}
        \asegura{|c.elems|=|C_0.elems|}
        \asegura{\forall i\in c.elems: (i=k\yLuego c.elems[k]=C_0.elems[k]+1)\lor(i\not=k \yLuego c.elems[k]=C_0.elems[k])}
    \end{proc}
    \begin{proc}{numeroEventos}{\In c: \nt, \In k: T}{\ent}
        \requiere{k\in c.elems}
        \asegura{res=c.elems[k]}
    \end{proc}
\end{tad}
\pagebreak
\\Contador con historial:
{\color{yellow}
\begin{tad}{ContadoresHistorial}
    \obs{elems}{dict\smm{T,\ent}}
    \obs{tiempo}{\ent}
    \begin{proc}{nuevosContadores}{\In l: \TLista{T},\In fecha: \ent}{\nt}
        \requiere{true}
        \asegura{|c.elems|=|l||}
        \asegura{\forall e \in l: e\in res.elems \yLuego res.elems[e]=0 \land res.tiempo=fecha}
    \end{proc}
    \begin{proc}{aumentarContador}{\Inout c: \nt, \In k: T, \In fecha: \ent}{}
        \requiere{c=C_0}
        \requiere{k\in c.elems}
        \asegura{|c.elems|=|C_0.elems|}
        \asegura{\forall i\in c.elems: (i=k\yLuego c.elems[k]=C_0.elems[k]+1 \land c.tiempo=fecha)\lor(i\not=k \yLuego c.elems[k]=C_0.elems[k])}
    \end{proc}
    \begin{proc}{numeroEventos}{\In c: \nt, \In k: T}{\ent}
        \requiere{k\in c.elems}
        \asegura{res=c.elems[k]}
    \end{proc}
\end{tad}}
\\\rojo{- muy confuso, lo dejo para despues}
\ejercicio{9}{Supongamos que queremos utilizar el TAD contador en un sistema que procesa millones de eventos por segundo
y no damos abasto para procesar todos los eventos. Una posible solución es hacer sampling: en lugar de contar cada evento,
contamos un porcentaje (configurable) de ellos, por ejemplo, un 1 \%. Es decir que de todas las llamadas al proc incrementar,
sólo el 1 \% de ellas efectivamente incrementa el contador.}
\rojo{En mi opinion es incorrecta porque no se deberia llamar a incrementar y ahi decir si se incrementa o no, en su lugar deberian
 seleccionarse solo el 1\% de los eventos y que esos eventos llamen a incrementar.\\ 
 Entonces lo que haria seria dejar la especificacion normal de incrementar y descartar los eventos antes.
 Por ejemplo con un $\forall i \in \ent, 0\leq i, i mod 99=0$ (porque son millones por segundo, en otro caso no se como se haria.)
 \\
 \\
 tl;dr: No sé}
 \ejercicio{10}{Un caché es una capa de almacenamiento de datos de alta velocidad que almacena un subconjunto de datos,
 normalmente transitorios, de modo que las solicitudes futuras de dichos datos se atienden con mayor rapidez que si se debe
 acceder a los datos desde la ubicación de almacenamiento principal. El almacenamiento en caché permite reutilizar de forma
 eficaz los datos recuperados o procesados anteriormente.
 Esta estructura comunmente tiene una interface de diccionario: guarda valores asociados a claves, con la diferencia de
 que los datos almacenados en un cache pueden desaparecer en cualquier momento, en función de diferentes criterios.
 Especificar caches genéricos (con claves de tipo K y valores de tipo V) que respeten las operaciones indicadas y las
 siguientes políticas de borrado automático. Si necesita conocer la fecha y hora actual, puede pasarla como parámetro a los
 procedimientos o bien puede asumir que existe una función externa horaActual() : Z que retorna la fecha y hora actual.
 Asuma que las fechas son números enteros (por ejemplo, la cantidad de segundos desde el 1 de enero de 1970).}
\begin{tad}{Cache\smm{K,V}}
    \begin{proc}{esta}{\In c:\nt}{\bool}\end{proc}
    \begin{proc}{obtener}{\In c:\nt, \In k: K}{V}\end{proc}
    \begin{proc}{definir}{\Inout c:\nt, \In k: K}{V}\end{proc}
\end{tad}
\pagebreak\\
Fifo o first-in-first-out
\\El cache tiene una capacidad maxima. si se alcanza esa capacidad maxima se borra automaticamente la clave que fue definida por primera vez hace mas tiempo.
{\color{olive}\begin{tad}{Cache\smm{K,V}}
    \obs{elems}{struct\smm{dict:dict\smm{K,V}, tiempo:\ent}}
    \obs{llena}{\bool}
    \begin{proc}{esta}{\In c:\nt, \In k: K}{\bool}
        \requiere{true}
        \asegura{res=true \sii k\in c.elems_{dict}}
    \end{proc}
    \begin{proc}{obtener}{\In c:\nt, \In k: K}{V}
        \requiere{k\in c.elems_{dict}}
        \asegura{res=c.elems_{dict}[k]}
    \end{proc}
    \begin{proc}{definir}{\Inout c:\nt, \In k: K, \In v: V}{}
        \requiere{c=C_0}
        \asegura{(C_0.llena \yLuego (\exists e \in C_0.elems: e_{tiempo}=\\
        \hspace*{7em}min(C_0.elems_{tiempo}) \yLuego c.elems_{dict} = setKey(delKey(c.elems_{dict},e)),k,v)))\lor_L\\
        \hspace*{4.5em}(\lnot C_0.llena \yLuego setKey(c.elems_{dict}, k, v))}
        \asegura{res.elems_{tiempo}=fechaActual()}
    \end{proc}
\end{tad}
\\---------------Flashe una banda, mejor ni terminarlo y arrancar denuevo. 
(Defini mal mi observador y solo tengo una fecha jdlksjf)}
\begin{tad}{Cache\smm{K,V}}
    \obs{datos}{dict\smm{K,struct\slr{valor: V,tiempo: \ent}}}
    \obs{lleno}{\bool}
    \begin{proc}{esta}{\In c:\nt, \In k: K}{\bool}
        \asegura{res=true\sii k \in c.datos}
    \end{proc}
    \begin{proc}{obtener}{\In c:\nt, \In k: K}{V}
        \requiere{k \in c.datos}
        \asegura{res=c.datos[k]_{valor}}
    \end{proc}
    \begin{proc}{definir}{\Inout c:\nt, \In k: K, \In v: V}{}
        \requiere{c=C_0}
        \asegura{C_0.lleno\yLuego 
        \exists e\in C_0.datos: (\forall e' \in C_0.datos: e_{tiempo} \leq e'_{tiempo})\yLuego\\
         c.datos=setKey(delKey(C_0.datos,e),k,\slr{v, fechaActual()\ }\ )\ }
        \asegura{\lnot C_0.lleno \yLuego c.datos=setKey(C_0.datos, k, \slr{v, fechaActual()\ }\ )\ }
    \end{proc}
\end{tad}
LRU o last-recently-used
\begin{tad}{Cache\smm{K,V}}
    \obs{datos}{dict\smm{K,struct\slr{valor: V,tiempo: \ent}}}
    \obs{lleno}{\bool}
    \begin{proc}{esta}{\In c:\nt, \In k: K}{\bool}
        \asegura{res=true\sii k \in c.datos}
    \end{proc}
    \begin{proc}{obtener}{\Inout c:\nt, \In k: K}{V}
        \requiere{c=C_0}
        \requiere{k \in C_0.datos}
        \asegura{res=C_0.datos[k]_{valor}}
        \asegura{c.datos[k]_{tiempo}=fechaActual()}
    \end{proc}
    \begin{proc}{definir}{\Inout c:\nt, \In k: K, \In v: V}{}
        \requiere{c=C_0}
        \asegura{C_0.lleno\yLuego 
        \exists e\in C_0.datos: (\forall e' \in C_0.datos: e_{tiempo} \leq e'_{tiempo})\yLuego\\
         c.datos=setKey(delKey(C_0.datos,e),k,\slr{v, fechaActual()\ }\ )\ }
        \asegura{\lnot C_0.lleno \yLuego c.datos=setKey(C_0.datos, k, \slr{v, fechaActual()\ }\ )\ }
    \end{proc}
\end{tad}
\anotacion{Entiendo el concepto del TTL pero no veo como aplicar algo que deberia ser automatico
\\ a lenguaje especificacion. Sobre los primeros dos creo que estan bien ahora si.}
\ejercicio{12}{Queremos modelar el funcionamiento de un vivero. El vivero arranca su actividad sin ninguna planta y con
un monto inicial de dinero.
Las plantas las compramos en un mayorista que nos vende la cantidad que deseemos pero solamente de a una especie
por vez. Como vivimos en un país con inflación, cada vez que vamos a comprar tenemos un precio diferente para la misma
planta. Para poder comprar plantas tenemos que tener suficiente dinero disponible, ya que el mayorista no acepta fiarnos.
A cada planta le ponemos un precio de venta por unidad. Ese precio tenemos que poder cambiarlo todas las veces que
necesitemos. Para simplificar el problema, asumimos que las plantas las vendemos de a un ejemplar (cada venta involucra
un solo ejemplar de una única especie). Por supuesto que para poder hacer una venta tenemos que tener stock de esa planta
y tenemos que haberle fijado un precio previamente. Además, se quiere saber en todo momento cuál es el balance de caja,
es decir, el dinero que tiene disponible el vivero.}
\begin{tad}{vivero}
    \obs{dinero}{\float}
    \obs{plantas}{dict\smm{E,$\nat_0$}}
    \obs{catalogo}{dict\smm{E,\float}}
    \obs{preciosMayorista}{dict\smm{E,\float}}
    \begin{proc}{comprarPlantas}{\Inout v: \nt, \In especie: E, \In cantidad: \nat}{}
        \requiere{v=V_0}
        \requiere{V_0.dinero>V_0.preciosMayorista[especie]*cantidad}
        \asegura{especie\in V_0.plantas \yLuego v.plantas=setKey(V_0.plantas,especie,V_0.plantas[especie]+cantidad)}
        \asegura{especie\not\in V_0.plantas \yLuego v.plantas=setKey(V_0.plantas,especie,cantidad)}
        \asegura{v.dinero=V_0.dinero-V_0.preciosMayorista[especie]*cantidad}
        \asegura{especie\in V_0.catalogo \yLuego\\\hspace*{4em}v.catalogo=setKey(V_0.catalogo, especie, max(V_0.catalogo[especie],V_0.preciosMayorista[especie]))}
        \asegura{especie\not\in V_0.catalogo \yLuego v.catalogo=setKey(V_0.catalogo, especie, 0))}
        \asegura{v.preciosMayorista>V_0.preciosMayorista}
    \end{proc}
    \begin{proc}{ponerPrecio}{\Inout v: \nt, \In especie: E, \In precio: \float}{}
        \requiere{v=V_0}
        \asegura{v.catalogo=setKey(V_0.catalogo,especie,precio)}
        \asegura{v.preciosMayorista=V_0.preciosMayorista}
        \asegura{v.plantas=V_0.plantas}
        \asegura{v.dinero=V_0.dinero}
    \end{proc}
    \begin{proc}{venderPlanta}{\Inout v: \nt, \In especie: E}{}
        \requiere{v=V_0}
        \requiere{V_0.catalogo[especie]\not=0}
        \requiere{V_0.plantas[especie]>0}
        \asegura{v.dinero=V_0.dinero+V_0.catalogo[especie]}
        \asegura{v.plantas=setKey(V_0.plantas,especie,V_0.plantas[especie]-1)}
        \asegura{v.preciosMayorista=V_0.preciosMayorista}
        \asegura{v.catalogo=V_0.catalogo}
    \end{proc}
    \begin{proc}{consultarCaja}{\In vivero: \nt}{\float}
        \asegura{res=vivero.dinero}
    \end{proc}
    \anotacion{--- el resto de consultas son triviales}
\end{tad}
 \end{document}