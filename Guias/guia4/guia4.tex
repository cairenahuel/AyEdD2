\documentclass[10pt,a4paper]{article}
\usepackage{tabularx}
\usepackage{amssymb}
\usepackage{enumitem}
\usepackage[spanish,activeacute,es-tabla]{babel}
\usepackage[utf8]{inputenc}
\usepackage{ifthen}
\usepackage{listings}
\usepackage{dsfont}
\usepackage{subcaption}
\usepackage{amsmath}
\usepackage[top=1cm,bottom=2cm,left=1cm,right=1cm]{geometry}%
\usepackage{color}%
\usepackage{changepage}
\newcommand{\tocarEspacios}{%
	\addtolength{\leftskip}{3em}%
	\setlength{\parindent}{0em}%
}

% Especificacion de procs

\newcommand{\In}{\textsf{in }}
\newcommand{\Out}{\textsf{out }}
\newcommand{\Inout}{\textsf{inout }}

\newcommand{\encabezadoDeProc}[4]{%
	% Ponemos la palabrita problema en tt
	%  \noindent%
	{\normalfont\bfseries\ttfamily proc}%
	% Ponemos el nombre del problema
	\ %
	{\normalfont\ttfamily #2}%
	\
	% Ponemos los parametros
	(#3)%
	\ifthenelse{\equal{#4}{}}{}{%
		% Por ultimo, va el tipo del resultado
		\ : #4}
}

\newenvironment{proc}[4][res]{%
	
	% El parametro 1 (opcional) es el nombre del resultado
	% El parametro 2 es el nombre del problema
	% El parametro 3 son los parametros
	% El parametro 4 es el tipo del resultado
	% Preambulo del ambiente problema
	% Tenemos que definir los comandos requiere, asegura, modifica y aux
	\newcommand{\requiere}[2][]{%
		{\normalfont\bfseries\ttfamily requiere}%
		\ifthenelse{\equal{##1}{}}{}{\ {\normalfont\ttfamily ##1} :}\ %
		\{\ensuremath{##2}\}%
		{\normalfont\bfseries\,\par}%
	}
	\newcommand{\asegura}[2][]{%
		{\normalfont\bfseries\ttfamily asegura}%
		\ifthenelse{\equal{##1}{}}{}{\ {\normalfont\ttfamily ##1} :}\
		\{\ensuremath{##2}\}%
		{\normalfont\bfseries\,\par}%
	}
	\renewcommand{\aux}[4]{%
		{\normalfont\bfseries\ttfamily aux\ }%
		{\normalfont\ttfamily ##1}%
		\ifthenelse{\equal{##2}{}}{}{\ (##2)}\ : ##3\, = \ensuremath{##4}%
		{\normalfont\bfseries\,;\par}%
	}
	\renewcommand{\pred}[3]{%
		{\normalfont\bfseries\ttfamily pred }%
		{\normalfont\ttfamily ##1}%
		\ifthenelse{\equal{##2}{}}{}{\ (##2) }%
		\{%
		\begin{adjustwidth}{+5em}{}
			\ensuremath{##3}
		\end{adjustwidth}
		\}%
		{\normalfont\bfseries\,\par}%
	}
	
	\newcommand{\res}{#1}
	\vspace{1ex}
	\noindent
	\encabezadoDeProc{#1}{#2}{#3}{#4}
	% Abrimos la llave
	\par%
	\tocarEspacios
}
{
	% Cerramos la llave
	\vspace{1ex}
}

\newcommand{\aux}[4]{%
	{\normalfont\bfseries\ttfamily\noindent aux\ }%
	{\normalfont\ttfamily #1}%
	\ifthenelse{\equal{#2}{}}{}{\ (#2)}\ : #3\, = \ensuremath{#4}%
	{\normalfont\bfseries\,;\par}%
}

\newcommand{\pred}[3]{%
	{\normalfont\bfseries\ttfamily\noindent pred }%
	{\normalfont\ttfamily #1}%
	\ifthenelse{\equal{#2}{}}{}{\ (#2) }%
	\{%
	\begin{adjustwidth}{+2em}{}
		\ensuremath{#3}
	\end{adjustwidth}
	\}%
	{\normalfont\bfseries\,\par}%
}

% Tipos

\newcommand{\nat}{\ensuremath{\mathds{N}}}
\newcommand{\ent}{\ensuremath{\mathds{Z}}}
\newcommand{\float}{\ensuremath{\mathds{R}}}
\newcommand{\bool}{\ensuremath{\mathsf{Bool}}}
\newcommand{\cha}{\ensuremath{\mathsf{Char}}}
\newcommand{\str}{\ensuremath{\mathsf{String}}}

% Logica

\newcommand{\True}{\ensuremath{\mathrm{true}}}
\newcommand{\False}{\ensuremath{\mathrm{false}}}
\newcommand{\Then}{\ensuremath{\rightarrow}}
\newcommand{\Iff}{\ensuremath{\leftrightarrow}}
\newcommand{\implica}{\ensuremath{\longrightarrow}}
\newcommand{\IfThenElse}[3]{\ensuremath{\mathsf{if}\ #1\ \mathsf{then}\ #2\ \mathsf{else}\ #3\ \mathsf{fi}}}
\newcommand{\yLuego}{\land _L}
\newcommand{\oLuego}{\lor _L}
\newcommand{\implicaLuego}{\implica _L}

\newcommand{\cuantificador}[5]{%
	\ensuremath{(#2 #3: #4)\ (%
		\ifthenelse{\equal{#1}{unalinea}}{
			#5
		}{
			$ % exiting math mode
			\begin{adjustwidth}{+2em}{}
				$#5$%
			\end{adjustwidth}%
			$ % entering math mode
		}
		)}
}

\newcommand{\existe}[4][]{%
	\cuantificador{#1}{\exists}{#2}{#3}{#4}
}
\newcommand{\paraTodo}[4][]{%
	\cuantificador{#1}{\forall}{#2}{#3}{#4}
}

%listas

\newcommand{\TLista}[1]{\ensuremath{seq \langle #1\rangle}}
\newcommand{\lvacia}{\ensuremath{[\ ]}}
\newcommand{\lv}{\ensuremath{[\ ]}}
\newcommand{\longitud}[1]{\ensuremath{|#1|}}
\newcommand{\cons}[1]{\ensuremath{\mathsf{addFirst}}(#1)}
\newcommand{\indice}[1]{\ensuremath{\mathsf{indice}}(#1)}
\newcommand{\conc}[1]{\ensuremath{\mathsf{concat}}(#1)}
\newcommand{\cab}[1]{\ensuremath{\mathsf{head}}(#1)}
\newcommand{\cola}[1]{\ensuremath{\mathsf{tail}}(#1)}
\newcommand{\sub}[1]{\ensuremath{\mathsf{subseq}}(#1)}
\newcommand{\en}[1]{\ensuremath{\mathsf{en}}(#1)}
\newcommand{\cuenta}[2]{\mathsf{cuenta}\ensuremath{(#1, #2)}}
\newcommand{\suma}[1]{\mathsf{suma}(#1)}
\newcommand{\twodots}{\ensuremath{\mathrm{..}}}
\newcommand{\masmas}{\ensuremath{++}}
\newcommand{\matriz}[1]{\TLista{\TLista{#1}}}
\newcommand{\seqchar}{\TLista{\cha}}

\renewcommand{\lstlistingname}{Código}
\lstset{% general command to set parameter(s)
	language=Java,
	morekeywords={endif, endwhile, skip},
	basewidth={0.47em,0.40em},
	columns=fixed, fontadjust, resetmargins, xrightmargin=5pt, xleftmargin=15pt,
	flexiblecolumns=false, tabsize=4, breaklines, breakatwhitespace=false, extendedchars=true,
	numbers=left, numberstyle=\tiny, stepnumber=1, numbersep=9pt,
	frame=l, framesep=3pt,
	captionpos=b,
}
\usepackage{changepage}
\usepackage{xcolor}
\usepackage[T1]{fontenc}
\usepackage{listings}
\usepackage{color}
\usepackage{amssymb}

\definecolor{dkgreen}{rgb}{0,0.6,0}
\definecolor{gray}{rgb}{0.5,0.5,0.5}
\definecolor{mauve}{rgb}{0.58,0,0.82}

\lstset{frame=tb,
  language=Java,
  aboveskip=3mm,
  belowskip=3mm,
  showstringspaces=false,
  columns=flexible,
  basicstyle={\small\ttfamily},
  numbers=none,
  numberstyle=\tiny\color{gray},
  keywordstyle=\color{blue},
  commentstyle=\color{dkgreen},
  stringstyle=\color{mauve},
  breaklines=true,
  breakatwhitespace=true,
  tabsize=3
}
\newcommand{\noexiste}[4][]{%
	\cuantificador{#1}{\nexists}{#2}{#3}{#4}
}
\renewcommand*\ttdefault{txtt}
\renewcommand*\familydefault{\ttdefault} %% Only if the base font of the document is to be typewriter style
\newcommand{\tbft}[2]{\par\addvspace{\baselineskip}\textbf{#1}\hspace{0.35em}{#2}\\\par\addvspace{\baselineskip}}
\newcommand{\ejercicio}[2]{\par\addvspace{\baselineskip}\textbf{Ejercicio #1.}\hspace{0.35em}{#2}\\\par\addvspace{\baselineskip}}
% \ejercicio{NUMERO}{ENUNCIADO}
%   Devuelve
% Ejercicio NUMERO. ---- ENUNCIADO ----
%
%formato
\newcommand{\salto}[1]{\par\addvspace{#1}}
\newcommand{\rojo}[1]{{\color{red}#1}}
\newcommand{\anotacion}[2][red]{\salto{1ex}\noindent\texttt{\color{#1}#2}\salto{1ex}}
\newcommand{\anotacionns}[2][red]{\noindent\texttt{\color{#1}#2}}
\newcommand{\return}{\textbf{return }}
%
% si y solo si corto y largo
\newcommand{\sii}{\leftrightarrow}
\newcommand{\siiLargo}{\longleftrightarrow}
\newcommand{\slr}[1]{\ensuremath{\langle #1\rangle}}
\newcommand{\smm}[1]{\textless #1\textgreater}
\newcommand{\encabezadoTAD}[1]{\par\salto{1ex}\noindent TAD\ \ \normalfont\ttfamily#1 }
\newenvironment{tad}[1]{
\newcommand{\nombretad}{{\ttfamily#1}}
\newcommand{\nt}{\nombretad}
\newcommand{\obs}[2]{\par\noindent{\ttfamily obs} ##1: ##2\par}
\encabezadoTAD{#1}\{
    \begin{adjustwidth}{3em}{0em}}
{\end{adjustwidth}\par\}}
% \begin{tad}{nombre del tad}
%   AGREGAR UN OBSERVADOR
%     \obs{nombre observador}{tipo}
%     \nombretad <---------- DEVUELVE EL NOMBRE DEL TAD (du)
%
%   y aca se pueden usar todos los procs y cosas de la catedra
%
%     \end{tad}
\newcommand{\encabezadoImpl}[3]{\par\salto{1ex}\noindent \textbf{impl}\ \normalfont\ttfamily#1(#2):#3}
\newcommand{\asg}[2]{\salto{0em}\noindent{\ttfamily #1}:= #2\salto{0em}}
\newcommand{\asgns}[2]{\noindent{\ttfamily #1}:= #2}
\newenvironment{impl}[3]{\encabezadoImpl{#1}{#2}{#3}\{
\begin{adjustwidth}{3em}{0em}
}
{\end{adjustwidth}\par\}}
\newcommand{\encabezadoDesign}[2]{\par\salto{1ex}\noindent \textbf{modulo}\ {\normalfont\ttfamily #1} \textbf{implementa}{ \normalfont\ttfamily #2}}
\newenvironment{design}[2]{\encabezadoDesign{#1}{#2}\{
\newcommand{\var}[2]{\salto{0em}\noindent{\normalfont \bfseries var} \texttt{##1}: \texttt{##2}\salto{0em}}
\begin{adjustwidth}{3em}{0em}
}
{\end{adjustwidth}\}}
\newcommand{\ifthel}[3]{\salto{0ex}
\noindent{\normalfont \bfseries if}{ #1 }{\normalfont \bfseries then}\salto{0ex}
\noindent{\begin{adjustwidth}{1.5em}{0em}#2\end{adjustwidth}}\salto{0ex}
\noindent{\normalfont \bfseries else}\begin{adjustwidth}{1.5em}{0em}{#3}\end{adjustwidth}\salto{0ex}
\noindent{\bfseries end if}}\salto{0ex}
\newcommand{\while}[2]{
    \salto{0em}\noindent
    {\normalfont \bfseries while} \ensuremath{#1} \textbf{do}
    \begin{adjustwidth}{1.5em}{0em}{#2}\end{adjustwidth}\salto{0em}\noindent{\normalfont \bfseries end while}
}
\newcommand{\invrep}[2]{\salto{0em}\noindent{\bfseries pred} InvRep (#1)\{\salto{0em}\noindent\begin{adjustwidth}{2em}{0em}{#2}\end{adjustwidth}\salto{0em}\noindent\}}
\newcommand{\abs}[2]{\salto{0em}\noindent{\bfseries pred} Abs (#1)\{\salto{0em}\noindent\begin{adjustwidth}{2em}{0em}{#2}\end{adjustwidth}\salto{0em}\noindent\}}

\begin{document}
\section*{Especificacion de TADs}
\tbft{Ejercicio 1}{Especificar en forma completa el TAD \texttt{NumeroRacional} que incluya al menos las operaciones aritmeticas basicas}
\begin{tad}{NumeroRacional}
    \obs{n}{\ent}
    \obs{d}{\ent}
    \begin{proc}{suma}{\In a,b: \nombretad}{\nombretad}
        \requiere{true}
        \asegura{res=\frac{a.n*b.d+a.d*b.n}{b.d*a.d}}
    \end{proc}
    \begin{proc}{resta}{\In a,b: \nombretad}{\nombretad}
        \requiere{true}
        \asegura{res=\frac{a.n*b.d-a.d*b.n}{b.d*a.d}}
    \end{proc}
    \begin{proc}{multiplicacion}{\In a,b: \nombretad}{\nombretad}
        \requiere{true}
        \asegura{res=\frac{a.n*b.n}{b.d*a.d}}
    \end{proc}
    \begin{proc}{division}{\In a,b: \nombretad}{\nombretad}
        \requiere{b\not= 0}
        \asegura{res=\frac{a.n*b.d}{b.n*a.d}}
    \end{proc}
\end{tad}
\tbft{Ejercicio 2:}{Especificar TADs para las siguientes figuras geometricas. Tiene que contener las operaciones rotar, trasladar y escalar, y una mas propuesta por usted.}
\begin{itemize}
    \item Rectangulo (2D)
    \item Esfera (3D)
\end{itemize}
\begin{tad}{Rectangulo}
    \obs{alto}{\float}
    \obs{ancho}{\float}
    \obs{posicion}{$\langle\float\times\float\rangle$}
    \obs{angulo}{\float}
    \begin{proc}{nuevo\nombretad}{\In h: \float, \In w: float, \In x:\float,\In y:\float}{\nombretad}
        \requiere{w \geq 0 \land h \geq 0}
        \asegura{res.ancho=w \land res.alto=h \land res.posicion=(x,y) \land r.angulo=0}
    \end{proc}
    \begin{proc}{escalar}{\Inout r: \nombretad, mh:\float, mw:\float}{}
        \requiere{r=R_0}
        \asegura{(mw\geq0 \land mh \geq0) \yLuego r.posicion=R_0.posicion }
        \asegura{r.alto=|R_0.alto*mh| \land r.ancho=|R_0.ancho*mw|}
        \asegura{mw<0 \yLuego r.posicion_1=R_0.posicion_1+R_0.ancho*mw}
        \asegura{mh<0 \yLuego r.posicion_2=R_0.posicion_2+R_0.alto*mh}
    \end{proc}
    \begin{proc}{rotar}{\Inout r: \nombretad, \In rad: \float}{}
        \requiere{r=R_0}
        \asegura{r.angulo=R_0.angulo+rad}
    \end{proc}
    \begin{proc}{trasladar}{\Inout r: \nombretad, \In pos: $\langle x,y\rangle$}{}
        \requiere{r=R_0}
        \asegura{r.posicion=R_0.posicion+pos}
    \end{proc}
    \begin{proc}{area}{\In r: \nombretad}{\float}
        \asegura{res=r.alto \cdot r.ancho}        
    \end{proc}
\end{tad}
\pagebreak
\begin{tad}{Esfera}
    \obs{radio}{\float}
    \obs{centro}{$\langle \float\times\float\times\float \rangle$}
    %\vspace*{0.5em}
    %\noindent Por claridad de notacion $centro_x$, $centro_y$ y $centro_z$ seran\\usadas en lugar de sus variantes con numero.
    \obs{semieje}{$\langle \float\times\float\times\float \rangle$}
    \begin{proc}{nueva\nombretad}{\In r: \float, \In pos:$\langle \float\times\float\times\float \rangle$}{\nombretad}
        \asegura{res.radio={|r|}}
        \asegura{res.centro=pos}
    \end{proc}
    \begin{proc}{escalar}{\Inout esfera: \nombretad, \In e: \float}{}
        \requiere{esfera=esfera_0}
        \asegura{esfera.radio=ESFERA_0.radio*|e|}
    \end{proc}
    \begin{proc}{trasladar}{\Inout esfera: \nombretad, \In pos:$\langle x,y,z\rangle$}{}
        \requiere{esfera=ESFERA_0}
        \asegura{esfera.centro=ESFERA_0.centro+pos}
    \end{proc}
    \begin{proc}{rotar}{\Inout esfera: \nombretad, \In angulosrad: $\langle \alpha,\beta,\gamma \rangle$}{}
        \requiere{esfera=ESFERA_0}
        \asegura{esfera.semieje=ESFERA_0.semieje+angulosrad}
    \end{proc}
\end{tad}
\ejercicio{3}{Especifique el TAD DobleCola\slr{T}, en el que los elementos pueden insertarse al principio o al final y se eliminan por el medio.}
\begin{tad}{DobleCola}
    \obs{elems}{\TLista{T}}
    \begin{proc}{\nombretad Vacia}{}{\nt}
        \asegura{res=\slr{}}
    \end{proc}
    \begin{proc}{encolarFinal}{\Inout doblecola: \nt, \In e: T}{}
        \requiere{doblecola=DOBLECOLA_0}
        \asegura{doblecola.elems=concat(DOBLECOLA_0.elems, \langle e\rangle)}
    \end{proc}
    \begin{proc}{encolarInicio}{\Inout doblecola: \nt, \In e: T}{}
        \requiere{doblecola=DOBLECOLA_0}
        \asegura{doblecola.elems=concat(\langle e\rangle,DOBLECOLA_0.elems)}
    \end{proc}
    \begin{proc}{desencolar}{\Inout doblecola: \nt}{T}
        \requiere{doblecola=DOBLECOLA_0}
        \requiere{cola.elems\not=\slr{}}
        \asegura{res\not \in doblecola.elems}
        \asegura{res=DOBLECOLA_0.elems\Biggl[\Bigl\lceil{\frac{|DOBLECOLA_0|}{2}}\Bigr\rceil\Biggr]}
    \end{proc}
\end{tad}
\pagebreak
\ejercicio{4}{Especifique el TAD DiccionarioConHistoria. El mismo guarda para cada clave, todos los valores que se asociaron con la misma a lo largo del tiempo (en orden)}
\begin{tad}{DiccionarioConHistoria\<T,\TLista{K}\>}
    \obs{data}{dict<T,\TLista{K}>}
    \begin{proc}{\nt Vacio}{}{\nt}
        \asegura{res.data=\{\}}
    \end{proc}
    \begin{proc}{estaLaLlave}{\In dic:\nt, \In e: T}{\bool}
        \asegura{res=true \siiLargo e \in dic.data}
    \end{proc}
    \begin{proc}{definir}{\Inout dic: \nt,\In k: T, \In e: K}{}
        \requiere{DIC_0=dic \yLuego k \in RES_0.data}
        \asegura{dic.data[k][0]=setKey(DIC_0.data,k,concat(DIC_0.data[k],e))}
    \end{proc}
    \begin{proc}{consultarHistorial}{\In dic: \nt, \In k: T}{\TLista{K}}
        \asegura{res=res.data[k]}
    \end{proc}
    \begin{proc}{borrar}{\Inout dic \nt,\In k: T}{}
        \requiere{dic=DIC_0}
        \requiere{k\in dic.data}
        \asegura{dic.data=delKey(DIC_0, k)}
    \end{proc}
    \begin{proc}{tamaño}{\In dic: \nt}{\ent}
        \asegura{res=|dic.data|}
    \end{proc}
\end{tad}
\ejercicio{5}{Modifique el TAD ColaDePrioridad<T> para que, si hay muchos valores iguales al maximo, la operacion desapilarMax los desapile a todos.}
\begin{tad}{ColaDePrioridad\textless T\textgreater}
    ---Toda la implementacion normal de \nt ---
    \begin{proc}{desapilarMax}{\Inout cola:\nt}{\TLista{T}}
        \requiere{cola=COLA_0}
        \asegura{\not \exists\ e \in cola.data, \exists e' \in COLA_0.data: tienePriMax(COLA_0.data, e') \yLuego cola.data[e] = COLA_0.data[e']}
        \asegura{\forall e \in COLA_0.data, (tienePriMax(COLA_0.data, e) \yLuego e \in res)\lor (\lnot tienePriMax(COLA_0.data, e) \yLuego e\not\in res)}
    \end{proc}
\end{tad}
\end{document}