\documentclass[10pt,a4paper]{article}
\usepackage{tabularx}
\usepackage{amssymb}
\usepackage{enumitem}
\input{../AEDmacros.tex}
\usepackage{changepage}
\newcommand{\tbft}[2]{\par\addvspace{\baselineskip}\textbf{#1}\hspace{0.35em}{#2}\\\par\addvspace{\baselineskip}}
\newcommand{\ejercicio}[2]{\par\addvspace{\baselineskip}\textbf{Ejercicio #1.}\hspace{0.35em}{#2}\\\par\addvspace{\baselineskip}}
% \ejercicio{NUMERO}{ENUNCIADO}
%   Devuelve
% Ejercicio NUMERO. ---- ENUNCIADO ----
%
%salto de linea comodo
\newcommand{\salto}[1]{\par\addvspace{#1}}
%
% si y solo si corto y largo
\newcommand{\sii}{\leftrightarrow}
\newcommand{\siiLargo}{\longleftrightarrow}
%
% entorno TAD
\newenvironment{tad}[1]{
%
%   constructor del encabezado formateado
\newcommand{\encabezadoTAD}[1]{\salto{1ex}\par\noindent TAD\ \ \normalfont\ttfamily#1 }
%
\encabezadoTAD{#1}\{
    \begin{adjustwidth}{3em}{0em}
    \newcommand{\obs}[2]{\par\noindent{\ttfamily obs} ##1: ##2\par}
    \newcommand{\nombretad}{{\ttfamily#1}}}
{\end{adjustwidth}\par\}}
% \begin{tad}{nombre del tad}
%   AGREGAR UN OBSERVADOR
%     \obs{nombre observador}{tipo}
%     \nombretad <---------- DEVUELVE EL NOMBRE DEL TAD (du)
%
%   y aca se pueden usar todos los procs y cosas de la catedra
%
%     \end{tad}
\begin{document}
\section*{Especificacion de TADs}
\tbft{Ejercicio 1}{Especificar en forma completa el TAD \texttt{NumeroRacional} que incluya al menos las operaciones aritmeticas basicas}
\begin{tad}{NumeroRacional}
    \obs{n}{\ent}
    \obs{d}{\ent}
    \begin{proc}{suma}{\In a,b: \nombretad}{\nombretad}
        \requiere{true}
        \asegura{res=\frac{a.n*b.d+a.d*b.n}{b.d*a.d}}
    \end{proc}
    \begin{proc}{resta}{\In a,b: \nombretad}{\nombretad}
        \requiere{true}
        \asegura{res=\frac{a.n*b.d-a.d*b.n}{b.d*a.d}}
    \end{proc}
    \begin{proc}{multiplicacion}{\In a,b: \nombretad}{\nombretad}
        \requiere{true}
        \asegura{res=\frac{a.n*b.n}{b.d*a.d}}
    \end{proc}
    \begin{proc}{division}{\In a,b: \nombretad}{\nombretad}
        \requiere{b\not= 0}
        \asegura{res=\frac{a.n*b.d}{b.n*a.d}}
    \end{proc}
\end{tad}
\tbft{Ejercicio 2:}{Especificar TADs para las siguientes figuras geometricas. Tiene que contener las operaciones rotar, trasladar y escalar, y una mas propuesta por usted.}
\begin{itemize}
    \item Rectangulo (2D)
    \item Esfera (3D)
\end{itemize}
\begin{tad}{Rectangulo}
    \obs{alto}{\float}
    \obs{ancho}{\float}
    \obs{posicion}{$\langle\float\times\float\rangle$}
    \obs{angulo}{\float}
    \begin{proc}{nuevo\nombretad}{\In h: \float, \In w: float, \In x:\float,\In y:\float}{\nombretad}
        \requiere{w \geq 0 \land h \geq 0}
        \asegura{res.ancho=w \land res.alto=h \land res.posicion=(x,y) \land r.angulo=0}
    \end{proc}
    \begin{proc}{escalar}{\Inout r: \nombretad, mh:\float, mw:\float}{}
        \requiere{r=R_0}
        \asegura{(mw\geq0 \land mh \geq0) \yLuego r.posicion=R_0.posicion }
        \asegura{r.alto=|R_0.alto*mh| \land r.ancho=|R_0.ancho*mw|}
        \asegura{mw<0 \yLuego r.posicion_1=R_0.posicion_1+R_0.ancho*mw}
        \asegura{mh<0 \yLuego r.posicion_2=R_0.posicion_2+R_0.alto*mh}
    \end{proc}
    \begin{proc}{rotar}{\Inout r: \nombretad, \In rad: \float}{}
        \requiere{r=R_0}
        \asegura{r.angulo=R_0.angulo+rad}
    \end{proc}
    \begin{proc}{trasladar}{\Inout r: \nombretad, \In pos: $\langle x,y\rangle$}{}
        \requiere{r=R_0}
        \asegura{r.posicion=R_0.posicion+pos}
    \end{proc}
    \begin{proc}{area}{\In r: \nombretad}{\float}
        \asegura{res=r.alto \cdot r.ancho}        
    \end{proc}
\end{tad}
\pagebreak
\begin{tad}{Esfera}
    \obs{radio}{\float}
    \obs{centro}{$\langle \float\times\float\times\float \rangle$}
    %\vspace*{0.5em}
    %\noindent Por claridad de notacion $centro_x$, $centro_y$ y $centro_z$ seran\\usadas en lugar de sus variantes con numero.
    \obs{semieje}{$\langle \float\times\float\times\float \rangle$}
    \begin{proc}{nueva\nombretad}{\In r: \float, \In pos:$\langle \float\times\float\times\float \rangle$}{\nombretad}
        \asegura{res.radio={|r|}}
        \asegura{res.centro=pos}
    \end{proc}
    \begin{proc}{escalar}{\Inout esfera: \nombretad, \In e: \float}{}
        \requiere{esfera=esfera_0}
        \asegura{esfera.radio=ESFERA_0.radio*|e|}
    \end{proc}
    \begin{proc}{trasladar}{\Inout esfera: \nombretad, \In pos:$\langle x,y,z\rangle$}{}
        \requiere{esfera=ESFERA_0}
        \asegura{esfera.centro=ESFERA_0.centro+pos}
    \end{proc}
    \begin{proc}{rotar}{\Inout esfera: \nombretad, \In angulosrad: $\langle \alpha,\beta,\gamma \rangle$}{}
        \requiere{esfera=ESFERA_0}
        \asegura{esfera.semieje=ESFERA_0.semieje+angulosrad}
    \end{proc}
\end{tad}
\ejercicio{3}{Especifique el TAD DobleCola\slr{T}, en el que los elementos pueden insertarse al principio o al final y se eliminan por el medio.}
\begin{tad}{DobleCola}
    \obs{elems}{\TLista{T}}
    \begin{proc}{\nombretad Vacia}{}{\nt}
        \asegura{res=\slr{}}
    \end{proc}
    \begin{proc}{encolarFinal}{\Inout doblecola: \nt, \In e: T}{}
        \requiere{doblecola=DOBLECOLA_0}
        \asegura{doblecola.elems=concat(DOBLECOLA_0.elems, \langle e\rangle)}
    \end{proc}
    \begin{proc}{encolarInicio}{\Inout doblecola: \nt, \In e: T}{}
        \requiere{doblecola=DOBLECOLA_0}
        \asegura{doblecola.elems=concat(\langle e\rangle,DOBLECOLA_0.elems)}
    \end{proc}
    \begin{proc}{desencolar}{\Inout doblecola: \nt}{T}
        \requiere{doblecola=DOBLECOLA_0}
        \requiere{cola.elems\not=\slr{}}
        \asegura{res\not \in doblecola.elems}
        \asegura{res=DOBLECOLA_0.elems\Biggl[\Bigl\lceil{\frac{|DOBLECOLA_0|}{2}}\Bigr\rceil\Biggr]}
    \end{proc}
\end{tad}

\end{document}