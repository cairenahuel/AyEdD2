\documentclass[10pt,a4paper,2in]{article}
\usepackage{tabularx}
\usepackage{amssymb}
\usepackage{enumitem}
\usepackage[spanish,activeacute,es-tabla]{babel}
\usepackage[utf8]{inputenc}
\usepackage{ifthen}
\usepackage{listings}
\usepackage{dsfont}
\usepackage{subcaption}
\usepackage{amsmath}
\usepackage[top=1cm,bottom=2cm,left=1cm,right=1cm]{geometry}%
\usepackage{color}%
\usepackage{changepage}
\newcommand{\tocarEspacios}{%
	\addtolength{\leftskip}{3em}%
	\setlength{\parindent}{0em}%
}

% Especificacion de procs

\newcommand{\In}{\textsf{in }}
\newcommand{\Out}{\textsf{out }}
\newcommand{\Inout}{\textsf{inout }}

\newcommand{\encabezadoDeProc}[4]{%
	% Ponemos la palabrita problema en tt
	%  \noindent%
	{\normalfont\bfseries\ttfamily proc}%
	% Ponemos el nombre del problema
	\ %
	{\normalfont\ttfamily #2}%
	\
	% Ponemos los parametros
	(#3)%
	\ifthenelse{\equal{#4}{}}{}{%
		% Por ultimo, va el tipo del resultado
		\ : #4}
}

\newenvironment{proc}[4][res]{%
	
	% El parametro 1 (opcional) es el nombre del resultado
	% El parametro 2 es el nombre del problema
	% El parametro 3 son los parametros
	% El parametro 4 es el tipo del resultado
	% Preambulo del ambiente problema
	% Tenemos que definir los comandos requiere, asegura, modifica y aux
	\newcommand{\requiere}[2][]{%
		{\normalfont\bfseries\ttfamily requiere}%
		\ifthenelse{\equal{##1}{}}{}{\ {\normalfont\ttfamily ##1} :}\ %
		\{\ensuremath{##2}\}%
		{\normalfont\bfseries\,\par}%
	}
	\newcommand{\asegura}[2][]{%
		{\normalfont\bfseries\ttfamily asegura}%
		\ifthenelse{\equal{##1}{}}{}{\ {\normalfont\ttfamily ##1} :}\
		\{\ensuremath{##2}\}%
		{\normalfont\bfseries\,\par}%
	}
	\renewcommand{\aux}[4]{%
		{\normalfont\bfseries\ttfamily aux\ }%
		{\normalfont\ttfamily ##1}%
		\ifthenelse{\equal{##2}{}}{}{\ (##2)}\ : ##3\, = \ensuremath{##4}%
		{\normalfont\bfseries\,;\par}%
	}
	\renewcommand{\pred}[3]{%
		{\normalfont\bfseries\ttfamily pred }%
		{\normalfont\ttfamily ##1}%
		\ifthenelse{\equal{##2}{}}{}{\ (##2) }%
		\{%
		\begin{adjustwidth}{+5em}{}
			\ensuremath{##3}
		\end{adjustwidth}
		\}%
		{\normalfont\bfseries\,\par}%
	}
	
	\newcommand{\res}{#1}
	\vspace{1ex}
	\noindent
	\encabezadoDeProc{#1}{#2}{#3}{#4}
	% Abrimos la llave
	\par%
	\tocarEspacios
}
{
	% Cerramos la llave
	\vspace{1ex}
}

\newcommand{\aux}[4]{%
	{\normalfont\bfseries\ttfamily\noindent aux\ }%
	{\normalfont\ttfamily #1}%
	\ifthenelse{\equal{#2}{}}{}{\ (#2)}\ : #3\, = \ensuremath{#4}%
	{\normalfont\bfseries\,;\par}%
}

\newcommand{\pred}[3]{%
	{\normalfont\bfseries\ttfamily\noindent pred }%
	{\normalfont\ttfamily #1}%
	\ifthenelse{\equal{#2}{}}{}{\ (#2) }%
	\{%
	\begin{adjustwidth}{+2em}{}
		\ensuremath{#3}
	\end{adjustwidth}
	\}%
	{\normalfont\bfseries\,\par}%
}

% Tipos

\newcommand{\nat}{\ensuremath{\mathds{N}}}
\newcommand{\ent}{\ensuremath{\mathds{Z}}}
\newcommand{\float}{\ensuremath{\mathds{R}}}
\newcommand{\bool}{\ensuremath{\mathsf{Bool}}}
\newcommand{\cha}{\ensuremath{\mathsf{Char}}}
\newcommand{\str}{\ensuremath{\mathsf{String}}}

% Logica

\newcommand{\True}{\ensuremath{\mathrm{true}}}
\newcommand{\False}{\ensuremath{\mathrm{false}}}
\newcommand{\Then}{\ensuremath{\rightarrow}}
\newcommand{\Iff}{\ensuremath{\leftrightarrow}}
\newcommand{\implica}{\ensuremath{\longrightarrow}}
\newcommand{\IfThenElse}[3]{\ensuremath{\mathsf{if}\ #1\ \mathsf{then}\ #2\ \mathsf{else}\ #3\ \mathsf{fi}}}
\newcommand{\yLuego}{\land _L}
\newcommand{\oLuego}{\lor _L}
\newcommand{\implicaLuego}{\implica _L}

\newcommand{\cuantificador}[5]{%
	\ensuremath{(#2 #3: #4)\ (%
		\ifthenelse{\equal{#1}{unalinea}}{
			#5
		}{
			$ % exiting math mode
			\begin{adjustwidth}{+2em}{}
				$#5$%
			\end{adjustwidth}%
			$ % entering math mode
		}
		)}
}

\newcommand{\existe}[4][]{%
	\cuantificador{#1}{\exists}{#2}{#3}{#4}
}
\newcommand{\paraTodo}[4][]{%
	\cuantificador{#1}{\forall}{#2}{#3}{#4}
}

%listas

\newcommand{\TLista}[1]{\ensuremath{seq \langle #1\rangle}}
\newcommand{\lvacia}{\ensuremath{[\ ]}}
\newcommand{\lv}{\ensuremath{[\ ]}}
\newcommand{\longitud}[1]{\ensuremath{|#1|}}
\newcommand{\cons}[1]{\ensuremath{\mathsf{addFirst}}(#1)}
\newcommand{\indice}[1]{\ensuremath{\mathsf{indice}}(#1)}
\newcommand{\conc}[1]{\ensuremath{\mathsf{concat}}(#1)}
\newcommand{\cab}[1]{\ensuremath{\mathsf{head}}(#1)}
\newcommand{\cola}[1]{\ensuremath{\mathsf{tail}}(#1)}
\newcommand{\sub}[1]{\ensuremath{\mathsf{subseq}}(#1)}
\newcommand{\en}[1]{\ensuremath{\mathsf{en}}(#1)}
\newcommand{\cuenta}[2]{\mathsf{cuenta}\ensuremath{(#1, #2)}}
\newcommand{\suma}[1]{\mathsf{suma}(#1)}
\newcommand{\twodots}{\ensuremath{\mathrm{..}}}
\newcommand{\masmas}{\ensuremath{++}}
\newcommand{\matriz}[1]{\TLista{\TLista{#1}}}
\newcommand{\seqchar}{\TLista{\cha}}

\renewcommand{\lstlistingname}{Código}
\lstset{% general command to set parameter(s)
	language=Java,
	morekeywords={endif, endwhile, skip},
	basewidth={0.47em,0.40em},
	columns=fixed, fontadjust, resetmargins, xrightmargin=5pt, xleftmargin=15pt,
	flexiblecolumns=false, tabsize=4, breaklines, breakatwhitespace=false, extendedchars=true,
	numbers=left, numberstyle=\tiny, stepnumber=1, numbersep=9pt,
	frame=l, framesep=3pt,
	captionpos=b,
}
\usepackage{changepage}
\usepackage{xcolor}
\usepackage[T1]{fontenc}
\usepackage{listings}
\usepackage{color}
\usepackage{amssymb}

\definecolor{dkgreen}{rgb}{0,0.6,0}
\definecolor{gray}{rgb}{0.5,0.5,0.5}
\definecolor{mauve}{rgb}{0.58,0,0.82}

\lstset{frame=tb,
  language=Java,
  aboveskip=3mm,
  belowskip=3mm,
  showstringspaces=false,
  columns=flexible,
  basicstyle={\small\ttfamily},
  numbers=none,
  numberstyle=\tiny\color{gray},
  keywordstyle=\color{blue},
  commentstyle=\color{dkgreen},
  stringstyle=\color{mauve},
  breaklines=true,
  breakatwhitespace=true,
  tabsize=3
}
\newcommand{\noexiste}[4][]{%
	\cuantificador{#1}{\nexists}{#2}{#3}{#4}
}
\renewcommand*\ttdefault{txtt}
\renewcommand*\familydefault{\ttdefault} %% Only if the base font of the document is to be typewriter style
\newcommand{\tbft}[2]{\par\addvspace{\baselineskip}\textbf{#1}\hspace{0.35em}{#2}\\\par\addvspace{\baselineskip}}
\newcommand{\ejercicio}[2]{\par\addvspace{\baselineskip}\textbf{Ejercicio #1.}\hspace{0.35em}{#2}\\\par\addvspace{\baselineskip}}
% \ejercicio{NUMERO}{ENUNCIADO}
%   Devuelve
% Ejercicio NUMERO. ---- ENUNCIADO ----
%
%formato
\newcommand{\salto}[1]{\par\addvspace{#1}}
\newcommand{\rojo}[1]{{\color{red}#1}}
\newcommand{\anotacion}[2][red]{\salto{1ex}\noindent\texttt{\color{#1}#2}\salto{1ex}}
\newcommand{\anotacionns}[2][red]{\noindent\texttt{\color{#1}#2}}
\newcommand{\return}{\textbf{return }}
%
% si y solo si corto y largo
\newcommand{\sii}{\leftrightarrow}
\newcommand{\siiLargo}{\longleftrightarrow}
\newcommand{\slr}[1]{\ensuremath{\langle #1\rangle}}
\newcommand{\smm}[1]{\textless #1\textgreater}
\newcommand{\encabezadoTAD}[1]{\par\salto{1ex}\noindent TAD\ \ \normalfont\ttfamily#1 }
\newenvironment{tad}[1]{
\newcommand{\nombretad}{{\ttfamily#1}}
\newcommand{\nt}{\nombretad}
\newcommand{\obs}[2]{\par\noindent{\ttfamily obs} ##1: ##2\par}
\encabezadoTAD{#1}\{
    \begin{adjustwidth}{3em}{0em}}
{\end{adjustwidth}\par\}}
% \begin{tad}{nombre del tad}
%   AGREGAR UN OBSERVADOR
%     \obs{nombre observador}{tipo}
%     \nombretad <---------- DEVUELVE EL NOMBRE DEL TAD (du)
%
%   y aca se pueden usar todos los procs y cosas de la catedra
%
%     \end{tad}
\newcommand{\encabezadoImpl}[3]{\par\salto{1ex}\noindent \textbf{impl}\ \normalfont\ttfamily#1(#2):#3}
\newcommand{\asg}[2]{\salto{0em}\noindent{\ttfamily #1}:= #2\salto{0em}}
\newcommand{\asgns}[2]{\noindent{\ttfamily #1}:= #2}
\newenvironment{impl}[3]{\encabezadoImpl{#1}{#2}{#3}\{
\begin{adjustwidth}{3em}{0em}
}
{\end{adjustwidth}\par\}}
\newcommand{\encabezadoDesign}[2]{\par\salto{1ex}\noindent \textbf{modulo}\ {\normalfont\ttfamily #1} \textbf{implementa}{ \normalfont\ttfamily #2}}
\newenvironment{design}[2]{\encabezadoDesign{#1}{#2}\{
\newcommand{\var}[2]{\salto{0em}\noindent{\normalfont \bfseries var} \texttt{##1}: \texttt{##2}\salto{0em}}
\begin{adjustwidth}{3em}{0em}
}
{\end{adjustwidth}\}}
\newcommand{\ifthel}[3]{\salto{0ex}
\noindent{\normalfont \bfseries if}{ #1 }{\normalfont \bfseries then}\salto{0ex}
\noindent{\begin{adjustwidth}{1.5em}{0em}#2\end{adjustwidth}}\salto{0ex}
\noindent{\normalfont \bfseries else}\begin{adjustwidth}{1.5em}{0em}{#3}\end{adjustwidth}\salto{0ex}
\noindent{\bfseries end if}}\salto{0ex}
\newcommand{\while}[2]{
    \salto{0em}\noindent
    {\normalfont \bfseries while} \ensuremath{#1} \textbf{do}
    \begin{adjustwidth}{1.5em}{0em}{#2}\end{adjustwidth}\salto{0em}\noindent{\normalfont \bfseries end while}
}
\newcommand{\invrep}[2]{\salto{0em}\noindent{\bfseries pred} InvRep (#1)\{\salto{0em}\noindent\begin{adjustwidth}{2em}{0em}{#2}\end{adjustwidth}\salto{0em}\noindent\}}
\newcommand{\abs}[2]{\salto{0em}\noindent{\bfseries pred} Abs (#1)\{\salto{0em}\noindent\begin{adjustwidth}{2em}{0em}{#2}\end{adjustwidth}\salto{0em}\noindent\}}

\begin{document}
\section{Repaso Lógica Proposicional}
\hfill\\
\textbf{Ejercicio 1: }Determinar los valores de verdad de las siguientes proposiciones cuando el valor de verdad de a, b y c es verdadero y el de x e y es falso.\\
\begin{tabularx}{\textwidth}{|lX|c||lX|c|}
\hline
a)&$(\lnot x\lor b) $&$ True$& e) &$(\lnot(c\lor y))\leftrightarrow(\lnot c\land \lnot y)$&$True$\\
b)&$((c\lor (y\land a))\lor b)$&$ True$ & f)&$((c\lor y)\land(a\lor b))$&$True$\\
c)&$\lnot (c\lor y))$&$False$& g)&$((c\lor y)\land(a\lor b))\leftrightarrow (c\lor (y\land a)\lor b)$&$True$\\
d)&$\lnot(y\lor c)$&$False$&h) &$(\lnot c\land \lnot y)$&$False$\\
\hline
\end{tabularx}
\tbft{Ejercicio 2: }{Considere la siguiente oración: “Si es mi cumpleaños o hay torta, entonces hay torta”.}
\begin{itemize}
\item Escribir usando lógica proposicional y realizar la tabla de verdad
\item Asumiendo que la oración es verdadera y hay una torta, qué se puede concluir?
\item Asumiendo que la oración es verdadera y no hay una torta, qué se puede concluir?
\item Suponiendo que la oración es mentira (es falsa), se puede concluir algo?
\end{itemize}
\texttt{p: Es mi cumpleaños q: Hay torta}\\
$p\lor q \implies q$\\
\par\addvspace{\baselineskip}
\begin{center}
\begin{tabular}{|c|c|c|}
\hline
p&q&$p\lor q \implies q$\\
V&V&V\\
V&F&V\\
F&V&V\\
F&F&V\\
\hline
\end{tabular}\\
\end{center}
\par\addvspace{\baselineskip}
\texttt{\\Si asumimos que la oración es verdadera y hay torta puede o no ser su cumpleaños.\\
Si asumimos que la oración es falsa y no hay una torta, llegamos a un absurdo y por lo tanto la oración es verdadera.\\
La oración siempre es verdadera por lo tanto no se puede concluir nada.\\
(No estoy del todo seguro de eso ultimo.}
\tbft{Ejercicio 3:}{* Usando reglas de equivalencia (conmutatividad, asociatividad, De Morgan, etc) determinar si los siguientes pares de fórmulas son equivalencias. Indicar en cada paso qué regla se utilizó.}
\begin{enumerate}[label=\alph*]
\item $(p\lor q)\land (p \lor r)$ y $\lnot p \implies (q\land r)$
\\Distributiva $\rightarrow p\lor (q\land r)$ y $\lnot p \implies (q\land r)$
\\Implicacion $\rightarrow p\lor (q\land r)$ y $\lnot(\lnot p) \lor (q\land r)$
\\Simplificando $\rightarrow p\lor (q\land r)$ y $p \lor (q\land r)$
\\$\therefore$ ambas expreciones son equivalentes.
\item $\lnot(\lnot p)\implies (\lnot(\lnot p\land \lnot q))$ y q
\\Negacion de una negacion y D'Morgan $\rightarrow p \implies  (p \lor q)$
\\Implicacion $\rightarrow \lnot p \lor ( p\lor q)$
\\Asociando y complementando $\rightarrow True \lor q$
\\Dominacion $\rightarrow True$
\\$\therefore$ $q\not\equiv True$
\item $((True\land p)\land(\lnot p \lor False))\implies \lnot(\lnot p \lor q)$ y $p\land \lnot q$
\\Implicacion y simplificando $\rightarrow \lnot((True\land p)\land(\lnot p \lor False))\lor (p \land \lnot q)$ y $p\land \lnot q$
\\Simplificando $\rightarrow (\lnot(True\land p)\lor \lnot(\lnot p \lor False))\lor (p \land \lnot q)$ y $p\land \lnot q$
\\Simplificando $\rightarrow ((False\lor \lnot p)\lor (p \land True))\lor (p \land \lnot q)$ y $p\land \lnot q$
\\Complejizando $\rightarrow ((((False\lor \lnot p) \lor p) \land ((False\lor \lnot p) \lor True)))\lor (p \land \lnot q)$ y $p\land \lnot q$
\\Asociando y complementando $\rightarrow ((((False\lor True) \land ((False\lor \lnot p) \lor True)))\lor (p \land \lnot q)$ y $p\land \lnot q$
\\Dominacion $\rightarrow ((((True) \land ((True)))\lor (p \land \lnot q)$ y $p\land \lnot q$
\\Dominacion $\rightarrow (True) \lor (p \land \lnot q)$ y $p\land \lnot q$
\\Dominacion $\rightarrow True$ y $p\land \lnot q$
\\$\therefore True \not\equiv p\land \lnot q$
\\Hay un paso mal (me olvide de que $True \land p \equiv p$) y bueno cosas, termina siendo False pero sigue siendo $\not\equiv$.
\item $(p \lor (\lnot p \land q))$ y $\lnot p \implies q$
\\Distribuyendo $\rightarrow (p\lor \lnot p)\land(p\lor q$) y $\lnot p \implies q$
\\Complemento  $\rightarrow (True)\land(p\lor q)$ y $\lnot (\lnot p) \lor q$
\\Identidad y doble negacion $\rightarrow p\lor q$ y $p \lor q$
\\$\therefore$ $(p \lor (\lnot p \land q))\equiv (\lnot p \implies q)$
\item $p\implies(q\land \lnot(q\implies r))$ y $(\lnot p \lor q)\land(\lnot p\lor (q\land \lnot r))$
\\ Implicacion $\rightarrow p \implies (q\land \lnot(\lnot q \lor r))$ y $\lnot p \lor (q\land (q\land \lnot r)$
\\ Idempotencia y negacion $\rightarrow p \implies (q\land \lnot r)$ y $\lnot p \lor (q \land \lnot r)$
\\ Implicacion $\rightarrow p \implies (q\land \lnot r)$ y $p \implies (q \land \lnot r)$
\\ $\therefore (p \lor (\lnot p \land q)) \equiv \lnot p \implies q$
\end{enumerate}
\tbft{Ejercicio 4:}{}
\begin{tabularx}{\linewidth}{X|X}
    \begin{equation}
        \begin{gathered}
            (p\lor \lnot p)\\
            \True\\
            \text{Tautologia}
        \end{gathered}
    \end{equation}
    &
    \begin{equation}
        \begin{gathered}
            (p \land \lnot p)\\
            \False\\
            \text{Contradiccion}
        \end{gathered}
    \end{equation}\\
    \begin{equation}
        \begin{gathered}
            ((\lnot p \lor q) \iff (p\implies q))\\
            ((\lnot p \lor q) \iff (\lnot p\lor q))\\
            \text{Contingencia}
        \end{gathered}
    \end{equation}
    &
    \begin{equation}
        \begin{gathered}
            ((p\land q)\implies p)\\
            \lnot(p\land q)\lor p\\
            \lnot p \lor \lnot q \lor p\\
            q\\
            \text{Contingencia}
        \end{gathered}
    \end{equation}\\
    \begin{equation}
        \begin{gathered}
            ((\lnot p \lor q)\iff (p\implies q))\\
            ((\lnot p \lor q)\iff (\lnot p \lor q))\\
            \text{Tautologia}
        \end{gathered}
    \end{equation}
    &
    \begin{equation}
        \begin{gathered}
            ((p\rightarrow(q\rightarrow r))\rightarrow ((p\rightarrow q)\rightarrow (p\rightarrow r)))\\
            (\lnot p \lor \lnot q \lor r) \rightarrow ((\lnot p \lor q)\rightarrow (\lnot p \lor r))\\
            \lnot (p\land q\land \lnot r )\lor (p\land \lnot q)\lor (\lnot p \lor r)\\
            (\lnot p\lor \lnot q\lor r )\lor ((p\lor \lnot p)\land (\lnot p \lor \lnot q)\lor r)\\
            (\lnot p\lor \lnot q\lor r )\lor ((\True)\land (\lnot p \lor \lnot q)\lor r)\\
            (\lnot p\lor \lnot q\lor r )\lor ((\lnot p \lor \lnot q)\lor r)\\
            (\lnot p \lor \lnot q\lor r)\\
            q\land p \implies r
            \\\text{Contingencia}
            \\\text{\small(creo)}
        \end{gathered}
    \end{equation}
\end{tabularx}
\tbft{\\Ejercicio 7:}{Sean p, q y r tres variables de las que se sabe que\\
- p y q nunca estan indefinidas,\\
- r sii q es verdadera\\
Proponer una fórmula que nunca se indefina, utilizando siempre las tres variables y que sea verdadera si y solo si se cumple que:}
\begin{tabularx}{\linewidth}{XX}
        \texttt{a) Al menos una es verdadera.} & \texttt{d) Sólo p y q son verdaderas}\\
        \texttt{b) Ninguna es verdadera.} & \texttt{e) No todas al mismo tiempo son verdaderas.}\\
        \texttt{c) Exactamente una de las tres es verdadera.} & \texttt{f) r es verdadera.}\\
\end{tabularx}
\begin{equation*}
    a\longrightarrow p\lor_L q \lor_L r
\end{equation*}
\begin{equation*}
    b\longrightarrow p\land_L q \land_L r
\end{equation*}
\begin{equation*}
    c\longrightarrow p\lor_L q \lor_L r
\end{equation*}
\begin{equation*}
    d\longrightarrow (p\land_L q) \lor_L r
\end{equation*}
\begin{equation*}
    e\longrightarrow p\lor_L q \lor_L r
\end{equation*}
\begin{equation*}
    f\longrightarrow r\lor_L q \lor_L p
\end{equation*}
\pagebreak\\
\tbft{Ejercicio 8}{*Determinar, para cada aparición de variables, si dicha aparición se encuentra libre o ligada. En caso de estar
ligada, aclarar a qué cuantificador lo está. En los casos en que sea posible, proponer valores para las variables libres de modo
tal que las expresiones sean verdaderas.\\}
\begin{equation*}
    \paraTodo[unalinea]{x}{\ent}{0\leq x < n \implies x+y=z}
\end{equation*}
\texttt{x es una variable ligada y tambien libre, {n,y,z} son todas libres}

\begin{equation*}
    \paraTodo[unalinea]{x,y}{\ent}{0\leq x < n \land 0\leq y < m \implies x+y=z}
\end{equation*}
\texttt{{x,y} son variables ligadas y tambien libres, {n,m,z} son todas libres}
\begin{equation*}
    \paraTodo[unalinea]{j}{\ent}{0\leq j < 10 \implies j<0}
\end{equation*}
\texttt{j es una variable ligada y libre}
\begin{equation*}
    \paraTodo[unalinea]{j}{\ent}{(0\leq j \implies P(j))\land P(j)}
\end{equation*}
\texttt{j es una variable ligada y tambien libre}

\tbft{Ejercicio 9:}{* Sea P (x : $\ent$) y Q(x : $\ent$) dos predicados cualquiera. Explicar cuál es el error de traducción a fórmulas\\
de los siguientes enunciados. Dar un ejemplo en el cuál sucede el problema y luego corregirlo.}
\begin{itemize}
    \item "Todos los naturales menos a 10 cumplen p"\\
\begin{equation*}
    \paraTodo[unalinea]{i}{\nat}{0\leq i<10 \land P(i)}
\end{equation*}
\texttt{El cero no es un natural por lo que la traduccion deberia ser $0<i<10$}
    \item ".Algun natural menor a 10 cumple p"
\begin{equation*}
    \existe[unalinea]{i}{\ent}{0\leq i < 10 \implies P(i)}
\end{equation*}
\texttt{Al igual que el anterior i deberia ser un $\nat$ pero es $\ent$ y da lugar a que se indefina la proposion}
\item "Todos los naturales menores a 10 que cumplen p, cumplen q"
\begin{equation*}
    \paraTodo[unalinea]{x}{\ent}{0\leq x < 10 \implies P(x) \land Q(x)}
\end{equation*}
\end{itemize}

\end{document}