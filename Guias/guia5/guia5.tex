\documentclass[10pt,a4paper]{article}
\usepackage{changepage}
\newcommand{\tbft}[2]{\par\addvspace{\baselineskip}\textbf{#1}\hspace{0.35em}{#2}\\\par\addvspace{\baselineskip}}
\newcommand{\ejercicio}[2]{\par\addvspace{\baselineskip}\textbf{Ejercicio #1.}\hspace{0.35em}{#2}\\\par\addvspace{\baselineskip}}
% \ejercicio{NUMERO}{ENUNCIADO}
%   Devuelve
% Ejercicio NUMERO. ---- ENUNCIADO ----
%
%salto de linea comodo
\newcommand{\salto}[1]{\par\addvspace{#1}}
%
% si y solo si corto y largo
\newcommand{\sii}{\leftrightarrow}
\newcommand{\siiLargo}{\longleftrightarrow}
%
% entorno TAD
\newenvironment{tad}[1]{
%
%   constructor del encabezado formateado
\newcommand{\encabezadoTAD}[1]{\salto{1ex}\par\noindent TAD\ \ \normalfont\ttfamily#1 }
%
\encabezadoTAD{#1}\{
    \begin{adjustwidth}{3em}{0em}
    \newcommand{\obs}[2]{\par\noindent{\ttfamily obs} ##1: ##2\par}
    \newcommand{\nombretad}{{\ttfamily#1}}}
{\end{adjustwidth}\par\}}
% \begin{tad}{nombre del tad}
%   AGREGAR UN OBSERVADOR
%     \obs{nombre observador}{tipo}
%     \nombretad <---------- DEVUELVE EL NOMBRE DEL TAD (du)
%
%   y aca se pueden usar todos los procs y cosas de la catedra
%
%     \end{tad}
\input{../AEDmacros.tex}
\begin{document}
\ejercicio{1}{Quizas la forma mas simple de implementar un conjunto acotado sea mediante un array de tamaño fijo, utilizando la siguiente estructura:}
\begin{design}{ConjAcotadoArr\smm{T}}{ConjAcotado\smm{T}}
    \var{datos}{Array\smm{T}}
    \var{largo}{int}
\end{design}
{\salto{0em}En la variable \emph{datos} guardaremos los elementos. Como el tamaño del arreglo es fijo, 
necesitaremos otra variable, a la que llamaremos \emph{largo}, 
que nos indique cuantas casillas del arreglo \emph{datos} estan siendo usadas.}
{\par Con esta misma estrucutra tenemos dos opciones, permitir o no que haya repetidos.}
\begin{itemize}
    \item Escriba el invariante de representacion y la funcion de abstraccion para ambos casos.
    \item ¿Cual es la mas eficiente? En que casos usaria cada una.
    \item Escriba los algoritmos para las operaciones \emph{\bfseries agregar} un elemento y \emph{\bfseries sacar} un elemento para ambas versiones.
    \item Respecto a la operacion sacar, piense en un algoritmo que no requiera generar un nuevo arreglo para reemplazar datos, sino que lo resuelva modificando alguna de sus posiciones.
\end{itemize}
\begin{design}{ConjAcotadoArr\smm{T}}{ConjAcotado\smm{T}}
    \var{datos}{Array\smm{T}}
    \var{largo}{int}
    \anotacionns[blue]{Invariantes de representacion para con y sin repetidos respecticamente:}
    \invrep{c: ConjAcotadoArr\smm{T}}{
        $0\leq c.largo \leq c.datos.length$
        }
    \anotacionns[blue]{\\\small Acá tomo la decision de que los slots vacios de mi array sean los ultimos.}
    \invrep{c: ConjAcotadoArr\smm{T}}{
        $0\leq c.largo \leq c.datos.length \land \paraTodo[unalinea]{i,j}{\ent}{0\leq i,j < c.largo \yLuego c.datos[i]\not=c.datos[j]}$
    }
    \anotacionns[blue]{\\La funcion de abstraccion es la misma para ambos casos:}
    \abs{c': ConjAcotadoArr\smm{T}, c: ConjAcotado\smm{T}}{
        $c.cap=c'.array.length\yLuego \paraTodo[unalinea]{e}{T}{e\in c.elems \sii \existe[unalinea]{i}{\ent}{0\leq i <c'.largo \yLuego c'.data[i]=e}}$
    }
    \begin{impl}{agregar}{\Inout c: ConjAcotadoArr\smm{T},\In e: T}{}
        \ifthel{$c.largo\geq c.datos.length$}{return}{\ifthel{$c.pertenece(e)$}{return}{\asg{c.datos[c.largo]}{e}}}
    \end{impl}
    \begin{impl}{agregar}{\Inout c: ConjAcotadoArr\smm{T},\In e: T}{}
        \ifthel{c.largo<c.datos.length}{\asgns{c.datos[c.largo]}{e}}{skip}
    \end{impl}
    \pagebreak
    \begin{impl}{sacar}{\Inout c: ConjAcotadoArr\smm{T},\In e: T}{}
        \asgns{i}{0}
        \while{c.datos[i]\not=e\ $\&\&\ $ i<c.datos.length}{i++}
        \ifthel{c.datos[i]=e}{\asgns{c.datos[i]}{c.datos[c.largo]\\}
        \asgns{c.largo}{c.largo-1}}{skip}
    \end{impl}
\end{design}
\ejercicio{2}{¿Como implementaria una pila \emph{no acotada} utilizando arreglos?\\
Escriba la estructura propuesta, su invariante de representacion, funcion de abstraccion y las operaciones \textbf{apilar} y \textbf{desapilar}}
\begin{design}{PilaArray\smm{T}}{Pila\smm{T}}
    \var{datos}{Array\smm{T}}
    \var{usados}{int}
\invrep{c': PilaArray\smm{T}}{
    \anotacionns[blue]{No estoy seguro de necesitar un invariante de representacion.}
}
\abs{c': PilaArray\smm{T}, c: Pila\smm{T}}{
    $c'.datos.length=c.s.length \land \paraTodo[unalinea]{i}{\ent}{0\leq i < c'.datos.length}$
}
\begin{impl}{apilar}{\Inout c: PilaArray\smm{T},\In e: T}{}
    \ifthel{$c.usados\leq c.datos.length$}{
        \ifthel{$c.usados < c.datos.length$}{\asgns{c.datos[c.usados]}{e}}{}
    }{}
\end{impl}
\end{design}
\begin{design}{Nombre}{Implementa}
    \var{}{}
\invrep{args}{cuerpo}
\abs{args}{cuerpo}
\begin{impl}{nombre}{args}{retorna}

\end{impl}
\end{design}  
\end{document}